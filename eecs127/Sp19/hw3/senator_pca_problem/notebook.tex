
% Default to the notebook output style

    


% Inherit from the specified cell style.




    
\documentclass[11pt]{article}

    
    
    \usepackage[T1]{fontenc}
    % Nicer default font (+ math font) than Computer Modern for most use cases
    \usepackage{mathpazo}

    % Basic figure setup, for now with no caption control since it's done
    % automatically by Pandoc (which extracts ![](path) syntax from Markdown).
    \usepackage{graphicx}
    % We will generate all images so they have a width \maxwidth. This means
    % that they will get their normal width if they fit onto the page, but
    % are scaled down if they would overflow the margins.
    \makeatletter
    \def\maxwidth{\ifdim\Gin@nat@width>\linewidth\linewidth
    \else\Gin@nat@width\fi}
    \makeatother
    \let\Oldincludegraphics\includegraphics
    % Set max figure width to be 80% of text width, for now hardcoded.
    \renewcommand{\includegraphics}[1]{\Oldincludegraphics[width=.8\maxwidth]{#1}}
    % Ensure that by default, figures have no caption (until we provide a
    % proper Figure object with a Caption API and a way to capture that
    % in the conversion process - todo).
    \usepackage{caption}
    \DeclareCaptionLabelFormat{nolabel}{}
    \captionsetup{labelformat=nolabel}

    \usepackage{adjustbox} % Used to constrain images to a maximum size 
    \usepackage{xcolor} % Allow colors to be defined
    \usepackage{enumerate} % Needed for markdown enumerations to work
    \usepackage{geometry} % Used to adjust the document margins
    \usepackage{amsmath} % Equations
    \usepackage{amssymb} % Equations
    \usepackage{textcomp} % defines textquotesingle
    % Hack from http://tex.stackexchange.com/a/47451/13684:
    \AtBeginDocument{%
        \def\PYZsq{\textquotesingle}% Upright quotes in Pygmentized code
    }
    \usepackage{upquote} % Upright quotes for verbatim code
    \usepackage{eurosym} % defines \euro
    \usepackage[mathletters]{ucs} % Extended unicode (utf-8) support
    \usepackage[utf8x]{inputenc} % Allow utf-8 characters in the tex document
    \usepackage{fancyvrb} % verbatim replacement that allows latex
    \usepackage{grffile} % extends the file name processing of package graphics 
                         % to support a larger range 
    % The hyperref package gives us a pdf with properly built
    % internal navigation ('pdf bookmarks' for the table of contents,
    % internal cross-reference links, web links for URLs, etc.)
    \usepackage{hyperref}
    \usepackage{longtable} % longtable support required by pandoc >1.10
    \usepackage{booktabs}  % table support for pandoc > 1.12.2
    \usepackage[inline]{enumitem} % IRkernel/repr support (it uses the enumerate* environment)
    \usepackage[normalem]{ulem} % ulem is needed to support strikethroughs (\sout)
                                % normalem makes italics be italics, not underlines
    

    
    
    % Colors for the hyperref package
    \definecolor{urlcolor}{rgb}{0,.145,.698}
    \definecolor{linkcolor}{rgb}{.71,0.21,0.01}
    \definecolor{citecolor}{rgb}{.12,.54,.11}

    % ANSI colors
    \definecolor{ansi-black}{HTML}{3E424D}
    \definecolor{ansi-black-intense}{HTML}{282C36}
    \definecolor{ansi-red}{HTML}{E75C58}
    \definecolor{ansi-red-intense}{HTML}{B22B31}
    \definecolor{ansi-green}{HTML}{00A250}
    \definecolor{ansi-green-intense}{HTML}{007427}
    \definecolor{ansi-yellow}{HTML}{DDB62B}
    \definecolor{ansi-yellow-intense}{HTML}{B27D12}
    \definecolor{ansi-blue}{HTML}{208FFB}
    \definecolor{ansi-blue-intense}{HTML}{0065CA}
    \definecolor{ansi-magenta}{HTML}{D160C4}
    \definecolor{ansi-magenta-intense}{HTML}{A03196}
    \definecolor{ansi-cyan}{HTML}{60C6C8}
    \definecolor{ansi-cyan-intense}{HTML}{258F8F}
    \definecolor{ansi-white}{HTML}{C5C1B4}
    \definecolor{ansi-white-intense}{HTML}{A1A6B2}

    % commands and environments needed by pandoc snippets
    % extracted from the output of `pandoc -s`
    \providecommand{\tightlist}{%
      \setlength{\itemsep}{0pt}\setlength{\parskip}{0pt}}
    \DefineVerbatimEnvironment{Highlighting}{Verbatim}{commandchars=\\\{\}}
    % Add ',fontsize=\small' for more characters per line
    \newenvironment{Shaded}{}{}
    \newcommand{\KeywordTok}[1]{\textcolor[rgb]{0.00,0.44,0.13}{\textbf{{#1}}}}
    \newcommand{\DataTypeTok}[1]{\textcolor[rgb]{0.56,0.13,0.00}{{#1}}}
    \newcommand{\DecValTok}[1]{\textcolor[rgb]{0.25,0.63,0.44}{{#1}}}
    \newcommand{\BaseNTok}[1]{\textcolor[rgb]{0.25,0.63,0.44}{{#1}}}
    \newcommand{\FloatTok}[1]{\textcolor[rgb]{0.25,0.63,0.44}{{#1}}}
    \newcommand{\CharTok}[1]{\textcolor[rgb]{0.25,0.44,0.63}{{#1}}}
    \newcommand{\StringTok}[1]{\textcolor[rgb]{0.25,0.44,0.63}{{#1}}}
    \newcommand{\CommentTok}[1]{\textcolor[rgb]{0.38,0.63,0.69}{\textit{{#1}}}}
    \newcommand{\OtherTok}[1]{\textcolor[rgb]{0.00,0.44,0.13}{{#1}}}
    \newcommand{\AlertTok}[1]{\textcolor[rgb]{1.00,0.00,0.00}{\textbf{{#1}}}}
    \newcommand{\FunctionTok}[1]{\textcolor[rgb]{0.02,0.16,0.49}{{#1}}}
    \newcommand{\RegionMarkerTok}[1]{{#1}}
    \newcommand{\ErrorTok}[1]{\textcolor[rgb]{1.00,0.00,0.00}{\textbf{{#1}}}}
    \newcommand{\NormalTok}[1]{{#1}}
    
    % Additional commands for more recent versions of Pandoc
    \newcommand{\ConstantTok}[1]{\textcolor[rgb]{0.53,0.00,0.00}{{#1}}}
    \newcommand{\SpecialCharTok}[1]{\textcolor[rgb]{0.25,0.44,0.63}{{#1}}}
    \newcommand{\VerbatimStringTok}[1]{\textcolor[rgb]{0.25,0.44,0.63}{{#1}}}
    \newcommand{\SpecialStringTok}[1]{\textcolor[rgb]{0.73,0.40,0.53}{{#1}}}
    \newcommand{\ImportTok}[1]{{#1}}
    \newcommand{\DocumentationTok}[1]{\textcolor[rgb]{0.73,0.13,0.13}{\textit{{#1}}}}
    \newcommand{\AnnotationTok}[1]{\textcolor[rgb]{0.38,0.63,0.69}{\textbf{\textit{{#1}}}}}
    \newcommand{\CommentVarTok}[1]{\textcolor[rgb]{0.38,0.63,0.69}{\textbf{\textit{{#1}}}}}
    \newcommand{\VariableTok}[1]{\textcolor[rgb]{0.10,0.09,0.49}{{#1}}}
    \newcommand{\ControlFlowTok}[1]{\textcolor[rgb]{0.00,0.44,0.13}{\textbf{{#1}}}}
    \newcommand{\OperatorTok}[1]{\textcolor[rgb]{0.40,0.40,0.40}{{#1}}}
    \newcommand{\BuiltInTok}[1]{{#1}}
    \newcommand{\ExtensionTok}[1]{{#1}}
    \newcommand{\PreprocessorTok}[1]{\textcolor[rgb]{0.74,0.48,0.00}{{#1}}}
    \newcommand{\AttributeTok}[1]{\textcolor[rgb]{0.49,0.56,0.16}{{#1}}}
    \newcommand{\InformationTok}[1]{\textcolor[rgb]{0.38,0.63,0.69}{\textbf{\textit{{#1}}}}}
    \newcommand{\WarningTok}[1]{\textcolor[rgb]{0.38,0.63,0.69}{\textbf{\textit{{#1}}}}}
    
    
    % Define a nice break command that doesn't care if a line doesn't already
    % exist.
    \def\br{\hspace*{\fill} \\* }
    % Math Jax compatability definitions
    \def\gt{>}
    \def\lt{<}
    % Document parameters
    \title{senator\_pca\_qns}
    
    
    

    % Pygments definitions
    
\makeatletter
\def\PY@reset{\let\PY@it=\relax \let\PY@bf=\relax%
    \let\PY@ul=\relax \let\PY@tc=\relax%
    \let\PY@bc=\relax \let\PY@ff=\relax}
\def\PY@tok#1{\csname PY@tok@#1\endcsname}
\def\PY@toks#1+{\ifx\relax#1\empty\else%
    \PY@tok{#1}\expandafter\PY@toks\fi}
\def\PY@do#1{\PY@bc{\PY@tc{\PY@ul{%
    \PY@it{\PY@bf{\PY@ff{#1}}}}}}}
\def\PY#1#2{\PY@reset\PY@toks#1+\relax+\PY@do{#2}}

\expandafter\def\csname PY@tok@w\endcsname{\def\PY@tc##1{\textcolor[rgb]{0.73,0.73,0.73}{##1}}}
\expandafter\def\csname PY@tok@c\endcsname{\let\PY@it=\textit\def\PY@tc##1{\textcolor[rgb]{0.25,0.50,0.50}{##1}}}
\expandafter\def\csname PY@tok@cp\endcsname{\def\PY@tc##1{\textcolor[rgb]{0.74,0.48,0.00}{##1}}}
\expandafter\def\csname PY@tok@k\endcsname{\let\PY@bf=\textbf\def\PY@tc##1{\textcolor[rgb]{0.00,0.50,0.00}{##1}}}
\expandafter\def\csname PY@tok@kp\endcsname{\def\PY@tc##1{\textcolor[rgb]{0.00,0.50,0.00}{##1}}}
\expandafter\def\csname PY@tok@kt\endcsname{\def\PY@tc##1{\textcolor[rgb]{0.69,0.00,0.25}{##1}}}
\expandafter\def\csname PY@tok@o\endcsname{\def\PY@tc##1{\textcolor[rgb]{0.40,0.40,0.40}{##1}}}
\expandafter\def\csname PY@tok@ow\endcsname{\let\PY@bf=\textbf\def\PY@tc##1{\textcolor[rgb]{0.67,0.13,1.00}{##1}}}
\expandafter\def\csname PY@tok@nb\endcsname{\def\PY@tc##1{\textcolor[rgb]{0.00,0.50,0.00}{##1}}}
\expandafter\def\csname PY@tok@nf\endcsname{\def\PY@tc##1{\textcolor[rgb]{0.00,0.00,1.00}{##1}}}
\expandafter\def\csname PY@tok@nc\endcsname{\let\PY@bf=\textbf\def\PY@tc##1{\textcolor[rgb]{0.00,0.00,1.00}{##1}}}
\expandafter\def\csname PY@tok@nn\endcsname{\let\PY@bf=\textbf\def\PY@tc##1{\textcolor[rgb]{0.00,0.00,1.00}{##1}}}
\expandafter\def\csname PY@tok@ne\endcsname{\let\PY@bf=\textbf\def\PY@tc##1{\textcolor[rgb]{0.82,0.25,0.23}{##1}}}
\expandafter\def\csname PY@tok@nv\endcsname{\def\PY@tc##1{\textcolor[rgb]{0.10,0.09,0.49}{##1}}}
\expandafter\def\csname PY@tok@no\endcsname{\def\PY@tc##1{\textcolor[rgb]{0.53,0.00,0.00}{##1}}}
\expandafter\def\csname PY@tok@nl\endcsname{\def\PY@tc##1{\textcolor[rgb]{0.63,0.63,0.00}{##1}}}
\expandafter\def\csname PY@tok@ni\endcsname{\let\PY@bf=\textbf\def\PY@tc##1{\textcolor[rgb]{0.60,0.60,0.60}{##1}}}
\expandafter\def\csname PY@tok@na\endcsname{\def\PY@tc##1{\textcolor[rgb]{0.49,0.56,0.16}{##1}}}
\expandafter\def\csname PY@tok@nt\endcsname{\let\PY@bf=\textbf\def\PY@tc##1{\textcolor[rgb]{0.00,0.50,0.00}{##1}}}
\expandafter\def\csname PY@tok@nd\endcsname{\def\PY@tc##1{\textcolor[rgb]{0.67,0.13,1.00}{##1}}}
\expandafter\def\csname PY@tok@s\endcsname{\def\PY@tc##1{\textcolor[rgb]{0.73,0.13,0.13}{##1}}}
\expandafter\def\csname PY@tok@sd\endcsname{\let\PY@it=\textit\def\PY@tc##1{\textcolor[rgb]{0.73,0.13,0.13}{##1}}}
\expandafter\def\csname PY@tok@si\endcsname{\let\PY@bf=\textbf\def\PY@tc##1{\textcolor[rgb]{0.73,0.40,0.53}{##1}}}
\expandafter\def\csname PY@tok@se\endcsname{\let\PY@bf=\textbf\def\PY@tc##1{\textcolor[rgb]{0.73,0.40,0.13}{##1}}}
\expandafter\def\csname PY@tok@sr\endcsname{\def\PY@tc##1{\textcolor[rgb]{0.73,0.40,0.53}{##1}}}
\expandafter\def\csname PY@tok@ss\endcsname{\def\PY@tc##1{\textcolor[rgb]{0.10,0.09,0.49}{##1}}}
\expandafter\def\csname PY@tok@sx\endcsname{\def\PY@tc##1{\textcolor[rgb]{0.00,0.50,0.00}{##1}}}
\expandafter\def\csname PY@tok@m\endcsname{\def\PY@tc##1{\textcolor[rgb]{0.40,0.40,0.40}{##1}}}
\expandafter\def\csname PY@tok@gh\endcsname{\let\PY@bf=\textbf\def\PY@tc##1{\textcolor[rgb]{0.00,0.00,0.50}{##1}}}
\expandafter\def\csname PY@tok@gu\endcsname{\let\PY@bf=\textbf\def\PY@tc##1{\textcolor[rgb]{0.50,0.00,0.50}{##1}}}
\expandafter\def\csname PY@tok@gd\endcsname{\def\PY@tc##1{\textcolor[rgb]{0.63,0.00,0.00}{##1}}}
\expandafter\def\csname PY@tok@gi\endcsname{\def\PY@tc##1{\textcolor[rgb]{0.00,0.63,0.00}{##1}}}
\expandafter\def\csname PY@tok@gr\endcsname{\def\PY@tc##1{\textcolor[rgb]{1.00,0.00,0.00}{##1}}}
\expandafter\def\csname PY@tok@ge\endcsname{\let\PY@it=\textit}
\expandafter\def\csname PY@tok@gs\endcsname{\let\PY@bf=\textbf}
\expandafter\def\csname PY@tok@gp\endcsname{\let\PY@bf=\textbf\def\PY@tc##1{\textcolor[rgb]{0.00,0.00,0.50}{##1}}}
\expandafter\def\csname PY@tok@go\endcsname{\def\PY@tc##1{\textcolor[rgb]{0.53,0.53,0.53}{##1}}}
\expandafter\def\csname PY@tok@gt\endcsname{\def\PY@tc##1{\textcolor[rgb]{0.00,0.27,0.87}{##1}}}
\expandafter\def\csname PY@tok@err\endcsname{\def\PY@bc##1{\setlength{\fboxsep}{0pt}\fcolorbox[rgb]{1.00,0.00,0.00}{1,1,1}{\strut ##1}}}
\expandafter\def\csname PY@tok@kc\endcsname{\let\PY@bf=\textbf\def\PY@tc##1{\textcolor[rgb]{0.00,0.50,0.00}{##1}}}
\expandafter\def\csname PY@tok@kd\endcsname{\let\PY@bf=\textbf\def\PY@tc##1{\textcolor[rgb]{0.00,0.50,0.00}{##1}}}
\expandafter\def\csname PY@tok@kn\endcsname{\let\PY@bf=\textbf\def\PY@tc##1{\textcolor[rgb]{0.00,0.50,0.00}{##1}}}
\expandafter\def\csname PY@tok@kr\endcsname{\let\PY@bf=\textbf\def\PY@tc##1{\textcolor[rgb]{0.00,0.50,0.00}{##1}}}
\expandafter\def\csname PY@tok@bp\endcsname{\def\PY@tc##1{\textcolor[rgb]{0.00,0.50,0.00}{##1}}}
\expandafter\def\csname PY@tok@fm\endcsname{\def\PY@tc##1{\textcolor[rgb]{0.00,0.00,1.00}{##1}}}
\expandafter\def\csname PY@tok@vc\endcsname{\def\PY@tc##1{\textcolor[rgb]{0.10,0.09,0.49}{##1}}}
\expandafter\def\csname PY@tok@vg\endcsname{\def\PY@tc##1{\textcolor[rgb]{0.10,0.09,0.49}{##1}}}
\expandafter\def\csname PY@tok@vi\endcsname{\def\PY@tc##1{\textcolor[rgb]{0.10,0.09,0.49}{##1}}}
\expandafter\def\csname PY@tok@vm\endcsname{\def\PY@tc##1{\textcolor[rgb]{0.10,0.09,0.49}{##1}}}
\expandafter\def\csname PY@tok@sa\endcsname{\def\PY@tc##1{\textcolor[rgb]{0.73,0.13,0.13}{##1}}}
\expandafter\def\csname PY@tok@sb\endcsname{\def\PY@tc##1{\textcolor[rgb]{0.73,0.13,0.13}{##1}}}
\expandafter\def\csname PY@tok@sc\endcsname{\def\PY@tc##1{\textcolor[rgb]{0.73,0.13,0.13}{##1}}}
\expandafter\def\csname PY@tok@dl\endcsname{\def\PY@tc##1{\textcolor[rgb]{0.73,0.13,0.13}{##1}}}
\expandafter\def\csname PY@tok@s2\endcsname{\def\PY@tc##1{\textcolor[rgb]{0.73,0.13,0.13}{##1}}}
\expandafter\def\csname PY@tok@sh\endcsname{\def\PY@tc##1{\textcolor[rgb]{0.73,0.13,0.13}{##1}}}
\expandafter\def\csname PY@tok@s1\endcsname{\def\PY@tc##1{\textcolor[rgb]{0.73,0.13,0.13}{##1}}}
\expandafter\def\csname PY@tok@mb\endcsname{\def\PY@tc##1{\textcolor[rgb]{0.40,0.40,0.40}{##1}}}
\expandafter\def\csname PY@tok@mf\endcsname{\def\PY@tc##1{\textcolor[rgb]{0.40,0.40,0.40}{##1}}}
\expandafter\def\csname PY@tok@mh\endcsname{\def\PY@tc##1{\textcolor[rgb]{0.40,0.40,0.40}{##1}}}
\expandafter\def\csname PY@tok@mi\endcsname{\def\PY@tc##1{\textcolor[rgb]{0.40,0.40,0.40}{##1}}}
\expandafter\def\csname PY@tok@il\endcsname{\def\PY@tc##1{\textcolor[rgb]{0.40,0.40,0.40}{##1}}}
\expandafter\def\csname PY@tok@mo\endcsname{\def\PY@tc##1{\textcolor[rgb]{0.40,0.40,0.40}{##1}}}
\expandafter\def\csname PY@tok@ch\endcsname{\let\PY@it=\textit\def\PY@tc##1{\textcolor[rgb]{0.25,0.50,0.50}{##1}}}
\expandafter\def\csname PY@tok@cm\endcsname{\let\PY@it=\textit\def\PY@tc##1{\textcolor[rgb]{0.25,0.50,0.50}{##1}}}
\expandafter\def\csname PY@tok@cpf\endcsname{\let\PY@it=\textit\def\PY@tc##1{\textcolor[rgb]{0.25,0.50,0.50}{##1}}}
\expandafter\def\csname PY@tok@c1\endcsname{\let\PY@it=\textit\def\PY@tc##1{\textcolor[rgb]{0.25,0.50,0.50}{##1}}}
\expandafter\def\csname PY@tok@cs\endcsname{\let\PY@it=\textit\def\PY@tc##1{\textcolor[rgb]{0.25,0.50,0.50}{##1}}}

\def\PYZbs{\char`\\}
\def\PYZus{\char`\_}
\def\PYZob{\char`\{}
\def\PYZcb{\char`\}}
\def\PYZca{\char`\^}
\def\PYZam{\char`\&}
\def\PYZlt{\char`\<}
\def\PYZgt{\char`\>}
\def\PYZsh{\char`\#}
\def\PYZpc{\char`\%}
\def\PYZdl{\char`\$}
\def\PYZhy{\char`\-}
\def\PYZsq{\char`\'}
\def\PYZdq{\char`\"}
\def\PYZti{\char`\~}
% for compatibility with earlier versions
\def\PYZat{@}
\def\PYZlb{[}
\def\PYZrb{]}
\makeatother


    % Exact colors from NB
    \definecolor{incolor}{rgb}{0.0, 0.0, 0.5}
    \definecolor{outcolor}{rgb}{0.545, 0.0, 0.0}



    
    % Prevent overflowing lines due to hard-to-break entities
    \sloppy 
    % Setup hyperref package
    \hypersetup{
      breaklinks=true,  % so long urls are correctly broken across lines
      colorlinks=true,
      urlcolor=urlcolor,
      linkcolor=linkcolor,
      citecolor=citecolor,
      }
    % Slightly bigger margins than the latex defaults
    
    \geometry{verbose,tmargin=1in,bmargin=1in,lmargin=1in,rmargin=1in}
    
    

    \begin{document}
    
    
    \maketitle
    
    

    
    \hypertarget{pca-and-senate-voting-data}{%
\subsection{PCA and Senate Voting
Data}\label{pca-and-senate-voting-data}}

\hypertarget{places-where-you-have-to-write-code-are-marked-with-todo}{%
\subsubsection{Places where you have to write code are marked with
\#TODO}\label{places-where-you-have-to-write-code-are-marked-with-todo}}

    In this problem we are given, \(X\) the \(m \times n\) data matrix with
entries in \(\{-1,0,1\}\), where each row corresponds to a Senator, and
each column to a bill.

    \begin{Verbatim}[commandchars=\\\{\}]
{\color{incolor}In [{\color{incolor}1}]:} \PY{c+c1}{\PYZsh{} Import the necessary packages for data manipulation, computation and PCA }
        \PY{k+kn}{import} \PY{n+nn}{pandas} \PY{k}{as} \PY{n+nn}{pd}
        \PY{k+kn}{import} \PY{n+nn}{numpy} \PY{k}{as} \PY{n+nn}{np}
        \PY{k+kn}{import} \PY{n+nn}{scipy} \PY{k}{as} \PY{n+nn}{sp}
        \PY{k+kn}{import} \PY{n+nn}{matplotlib}\PY{n+nn}{.}\PY{n+nn}{pyplot} \PY{k}{as} \PY{n+nn}{plt}
        \PY{k+kn}{from} \PY{n+nn}{sklearn}\PY{n+nn}{.}\PY{n+nn}{decomposition} \PY{k}{import} \PY{n}{PCA}
        \PY{o}{\PYZpc{}}\PY{k}{matplotlib} inline
        
        \PY{n}{np}\PY{o}{.}\PY{n}{random}\PY{o}{.}\PY{n}{seed}\PY{p}{(}\PY{l+m+mi}{7}\PY{p}{)}
\end{Verbatim}


    \begin{Verbatim}[commandchars=\\\{\}]
{\color{incolor}In [{\color{incolor}2}]:} \PY{n}{senator\PYZus{}df} \PY{o}{=}  \PY{n}{pd}\PY{o}{.}\PY{n}{read\PYZus{}csv}\PY{p}{(}\PY{l+s+s1}{\PYZsq{}}\PY{l+s+s1}{senator\PYZus{}data\PYZus{}pca/data\PYZus{}matrix.csv}\PY{l+s+s1}{\PYZsq{}}\PY{p}{)}
        \PY{n}{affiliation\PYZus{}file} \PY{o}{=} \PY{n+nb}{open}\PY{p}{(}\PY{l+s+s2}{\PYZdq{}}\PY{l+s+s2}{senator\PYZus{}data\PYZus{}pca/politician\PYZus{}labels.txt}\PY{l+s+s2}{\PYZdq{}}\PY{p}{,} \PY{l+s+s2}{\PYZdq{}}\PY{l+s+s2}{r}\PY{l+s+s2}{\PYZdq{}}\PY{p}{)}
        \PY{n}{affiliations} \PY{o}{=} \PY{p}{[}\PY{n}{line}\PY{o}{.}\PY{n}{split}\PY{p}{(}\PY{l+s+s1}{\PYZsq{}}\PY{l+s+se}{\PYZbs{}n}\PY{l+s+s1}{\PYZsq{}}\PY{p}{)}\PY{p}{[}\PY{l+m+mi}{0}\PY{p}{]}\PY{o}{.}\PY{n}{split}\PY{p}{(}\PY{l+s+s1}{\PYZsq{}}\PY{l+s+s1}{ }\PY{l+s+s1}{\PYZsq{}}\PY{p}{)}\PY{p}{[}\PY{l+m+mi}{1}\PY{p}{]} \PY{k}{for} \PY{n}{line} \PY{o+ow}{in} \PY{n}{affiliation\PYZus{}file}\PY{o}{.}\PY{n}{readlines}\PY{p}{(}\PY{p}{)}\PY{p}{]}
        \PY{n}{X} \PY{o}{=} \PY{n}{np}\PY{o}{.}\PY{n}{array}\PY{p}{(}\PY{n}{senator\PYZus{}df}\PY{o}{.}\PY{n}{values}\PY{p}{[}\PY{p}{:}\PY{p}{,} \PY{l+m+mi}{3}\PY{p}{:}\PY{p}{]}\PY{o}{.}\PY{n}{T}\PY{p}{,} \PY{n}{dtype}\PY{o}{=}\PY{l+s+s1}{\PYZsq{}}\PY{l+s+s1}{float64}\PY{l+s+s1}{\PYZsq{}}\PY{p}{)} \PY{c+c1}{\PYZsh{}transpose to get senators as rows}
        \PY{n+nb}{print}\PY{p}{(}\PY{l+s+s2}{\PYZdq{}}\PY{l+s+s2}{X.shape: }\PY{l+s+s2}{\PYZdq{}}\PY{p}{,} \PY{n}{X}\PY{o}{.}\PY{n}{shape}\PY{p}{)}
        \PY{n}{n} \PY{o}{=} \PY{n}{X}\PY{o}{.}\PY{n}{shape}\PY{p}{[}\PY{l+m+mi}{0}\PY{p}{]} \PY{c+c1}{\PYZsh{}Number of senators}
        \PY{n}{m} \PY{o}{=} \PY{n}{X}\PY{o}{.}\PY{n}{shape}\PY{p}{[}\PY{l+m+mi}{1}\PY{p}{]} \PY{c+c1}{\PYZsh{}Number of bills}
\end{Verbatim}


    \begin{Verbatim}[commandchars=\\\{\}]
X.shape:  (100, 542)

    \end{Verbatim}

    \hypertarget{we-see-that-the-number-of-rows-n-is-the-number-of-senators-and-is-equal-to-100.-the-number-of-columns-m-is-the-number-of-bills-and-is-equal-to-542.}{%
\subparagraph{\texorpdfstring{We see that the number of rows \(n\), is
the number of senators and is equal to 100. The number of columns, \(m\)
is the number of bills and is equal to
542.}{We see that the number of rows n, is the number of senators and is equal to 100. The number of columns, m is the number of bills and is equal to 542.}}\label{we-see-that-the-number-of-rows-n-is-the-number-of-senators-and-is-equal-to-100.-the-number-of-columns-m-is-the-number-of-bills-and-is-equal-to-542.}}

    \begin{Verbatim}[commandchars=\\\{\}]
{\color{incolor}In [{\color{incolor}3}]:} \PY{n}{typical\PYZus{}row} \PY{o}{=} \PY{n}{X}\PY{p}{[}\PY{l+m+mi}{0}\PY{p}{,}\PY{p}{:}\PY{p}{]}
        \PY{n+nb}{print}\PY{p}{(}\PY{n}{typical\PYZus{}row}\PY{o}{.}\PY{n}{shape}\PY{p}{)}
        \PY{n+nb}{print}\PY{p}{(}\PY{n}{typical\PYZus{}row}\PY{p}{)}
\end{Verbatim}


    \begin{Verbatim}[commandchars=\\\{\}]
(542,)
[ 1.  1.  1. -1. -1.  1.  1.  1.  1. -1.  1. -1. -1.  1.  1. -1.  1.  1.
  1.  1.  1. -1.  1.  1.  1. -1.  1. -1.  1.  1.  1.  1.  1. -1.  1. -1.
 -1. -1. -1.  1.  1. -1. -1. -1. -1.  1.  1.  1. -1.  1.  1. -1.  1.  1.
 -1.  1.  1.  1.  1. -1.  1. -1. -1. -1.  1.  1.  1.  1.  1.  1.  1.  1.
  1.  1. -1.  0. -1.  1.  1.  1. -1. -1.  1.  1. -1. -1.  1.  1.  1. -1.
  1. -1.  1. -1.  1.  1. -1. -1. -1.  1.  1.  1. -1. -1. -1. -1. -1. -1.
  1. -1.  1.  1. -1. -1. -1.  1. -1.  1. -1.  1.  0.  0.  1.  1. -1.  1.
  1. -1.  1.  1. -1.  1. -1. -1.  1.  1.  1.  1.  0. -1. -1.  1.  1. -1.
  1.  1.  1.  1.  1.  0.  1.  0.  1.  1.  1.  1.  1.  1.  1. -1.  1.  1.
 -1.  1. -1.  1.  1.  1.  1.  1.  1.  1.  1.  1.  1.  1.  1.  1. -1.  1.
  1.  1.  1.  1.  1.  1.  1.  1.  1.  1.  1.  0.  1. -1. -1.  1.  1.  1.
  1.  1.  1.  1.  1.  1.  1.  1. -1.  1.  1.  1.  1.  1.  1.  1.  1. -1.
  1.  1.  0.  1.  0. -1.  1.  1.  1.  1.  1.  1. -1.  1.  1.  1.  1.  1.
  1.  1.  1.  1.  1.  1.  1.  1.  1.  1.  1.  1.  1.  1.  1.  1.  1.  1.
  1. -1.  1.  1.  1.  1.  1.  1.  1.  1.  1.  1.  1. -1.  1.  1.  1. -1.
  1.  1.  1.  1.  1.  1. -1. -1. -1.  1.  1. -1.  1. -1. -1.  1.  1.  1.
 -1.  1.  1.  1. -1.  1. -1.  1. -1. -1.  1. -1. -1.  1.  1.  1. -1.  1.
  1.  1.  1.  1. -1.  1.  1.  1.  1.  1.  1.  1.  1.  1.  1.  1. -1. -1.
  1. -1.  1. -1. -1.  1.  1.  1.  1.  1.  1.  1.  1.  1.  1.  1.  1.  1.
  1.  1.  1.  1.  1.  1.  1.  1.  1. -1.  1.  1.  1.  1.  1. -1.  1. -1.
  1.  1.  1.  1.  1.  1. -1.  1. -1. -1. -1. -1.  1.  1.  1.  1.  1.  1.
  1.  1.  1. -1.  1.  1.  1.  1.  1.  1.  1. -1. -1.  1. -1.  1.  1.  1.
  1.  1. -1.  1. -1.  1. -1.  1.  1. -1.  1.  1.  1.  1.  1.  1.  1.  1.
  1.  0.  1. -1.  1.  1.  1.  1.  1. -1. -1. -1.  1.  1.  0.  1.  1.  1.
  1.  1.  1.  1. -1. -1.  0.  0.  0.  0.  0.  0.  0.  1.  1. -1. -1.  1.
  1.  1.  1.  1.  1.  1.  1.  1.  1. -1.  1. -1.  1.  1.  1.  1. -1. -1.
  1.  1.  1.  1. -1. -1.  1.  1. -1.  1.  1.  1.  1.  1.  1.  1.  1.  1.
  1.  1.  1.  1.  1.  1.  1.  1.  1.  1.  1.  1.  1.  1. -1.  1.  1.  1.
 -1.  1.  1.  1.  1.  1.  1.  1.  1.  1.  1.  1.  1.  1.  1.  1.  1.  1.
  1.  1. -1. -1. -1.  1.  1.  1.  1. -1. -1.  1.  1. -1.  1.  1.  1.  1.
  1.  1.]

    \end{Verbatim}

    \hypertarget{a-typical-row-of-x-consists-of-entries--1-senator-voted-against-1senator-voted-for-and-0senator-abstained-for-each-bill.}{%
\subparagraph{\texorpdfstring{A typical row of \(X\) consists of entries
-1 (senator voted against), 1(senator voted for) and 0(senator
abstained) for each
bill.}{A typical row of X consists of entries -1 (senator voted against), 1(senator voted for) and 0(senator abstained) for each bill.}}\label{a-typical-row-of-x-consists-of-entries--1-senator-voted-against-1senator-voted-for-and-0senator-abstained-for-each-bill.}}

    \begin{Verbatim}[commandchars=\\\{\}]
{\color{incolor}In [{\color{incolor}4}]:} \PY{n}{typical\PYZus{}column} \PY{o}{=} \PY{n}{X}\PY{p}{[}\PY{p}{:}\PY{p}{,}\PY{l+m+mi}{0}\PY{p}{]}
        \PY{n+nb}{print}\PY{p}{(}\PY{n}{typical\PYZus{}column}\PY{o}{.}\PY{n}{shape}\PY{p}{)}
        \PY{n+nb}{print}\PY{p}{(}\PY{n}{typical\PYZus{}column}\PY{p}{)}
\end{Verbatim}


    \begin{Verbatim}[commandchars=\\\{\}]
(100,)
[ 1.  1.  1.  1.  1.  1.  1. -1.  1. -1.  1. -1.  1. -1. -1. -1.  1.  1.
 -1.  1.  1. -1.  1. -1.  1.  1.  1. -1. -1.  1.  1.  1. -1.  1.  1.  1.
 -1. -1. -1. -1.  1. -1. -1.  1.  1. -1. -1. -1. -1. -1.  1.  1. -1. -1.
  1.  1. -1. -1. -1. -1. -1.  1.  1.  1.  1.  1. -1. -1. -1.  1. -1. -1.
  1. -1. -1.  1.  1.  1. -1. -1. -1.  1.  1. -1.  1. -1.  1.  1.  1. -1.
 -1. -1. -1. -1.  1.  1.  1. -1. -1. -1.]

    \end{Verbatim}

    \hypertarget{a-typical-row-of-x-consists-of-entries-in--1-0-1-based-on-how-each-senator-voted-for-that-particular-bill.}{%
\subparagraph{\texorpdfstring{A typical row of \(X\) consists of entries
in \{-1, 0, 1\} based on how each senator voted for that particular
bill.}{A typical row of X consists of entries in \{-1, 0, 1\} based on how each senator voted for that particular bill.}}\label{a-typical-row-of-x-consists-of-entries-in--1-0-1-based-on-how-each-senator-voted-for-that-particular-bill.}}

    \begin{Verbatim}[commandchars=\\\{\}]
{\color{incolor}In [{\color{incolor}5}]:} \PY{n}{X\PYZus{}mean} \PY{o}{=} \PY{n}{np}\PY{o}{.}\PY{n}{mean}\PY{p}{(}\PY{n}{X}\PY{p}{,} \PY{n}{axis} \PY{o}{=} \PY{l+m+mi}{0}\PY{p}{)}
        \PY{n}{plt}\PY{o}{.}\PY{n}{plot}\PY{p}{(}\PY{n}{X\PYZus{}mean}\PY{p}{)}
        \PY{n}{plt}\PY{o}{.}\PY{n}{title}\PY{p}{(}\PY{l+s+s1}{\PYZsq{}}\PY{l+s+s1}{Mean of columns of X}\PY{l+s+s1}{\PYZsq{}}\PY{p}{)}
        \PY{n}{plt}\PY{o}{.}\PY{n}{show}\PY{p}{(}\PY{p}{)}
\end{Verbatim}


    \begin{center}
    \adjustimage{max size={0.9\linewidth}{0.9\paperheight}}{output_9_0.png}
    \end{center}
    { \hspace*{\fill} \\}
    
    \hypertarget{we-see-that-the-mean-of-the-columns-is-not-zero-so-we-center-the-data-by-subtracting-the-mean}{%
\subparagraph{We see that the mean of the columns is not zero so we
center the data by subtracting the
mean}\label{we-see-that-the-mean-of-the-columns-is-not-zero-so-we-center-the-data-by-subtracting-the-mean}}

    \begin{Verbatim}[commandchars=\\\{\}]
{\color{incolor}In [{\color{incolor}6}]:} \PY{n}{X\PYZus{}original} \PY{o}{=} \PY{n}{X}\PY{o}{.}\PY{n}{copy}\PY{p}{(}\PY{p}{)}
        \PY{n}{X} \PY{o}{=} \PY{n}{X} \PY{o}{\PYZhy{}} \PY{n}{np}\PY{o}{.}\PY{n}{mean}\PY{p}{(}\PY{n}{X}\PY{p}{,} \PY{n}{axis} \PY{o}{=} \PY{l+m+mi}{0}\PY{p}{)}
\end{Verbatim}


    \hypertarget{part-a-finding-a-unit-norm-m-vector-a-to-maximize-variance}{%
\subsubsection{\texorpdfstring{Part a) Finding a unit-norm \(m\)-vector
\(a\) to maximize
variance}{Part a) Finding a unit-norm m-vector a to maximize variance}}\label{part-a-finding-a-unit-norm-m-vector-a-to-maximize-variance}}

    \hypertarget{this-is-a-function-to-calculate-the-scores-fxa.}{%
\subparagraph{\texorpdfstring{This is a function to calculate the
scores,
\(f(X,a)\).}{This is a function to calculate the scores, f(X,a).}}\label{this-is-a-function-to-calculate-the-scores-fxa.}}

    \begin{Verbatim}[commandchars=\\\{\}]
{\color{incolor}In [{\color{incolor}7}]:} \PY{k}{def} \PY{n+nf}{f}\PY{p}{(}\PY{n}{X}\PY{p}{,} \PY{n}{a}\PY{p}{)}\PY{p}{:}
            \PY{k}{return} \PY{n}{np}\PY{o}{.}\PY{n}{matmul}\PY{p}{(}\PY{n}{X}\PY{p}{,}\PY{n}{a}\PY{p}{)}
\end{Verbatim}


    \hypertarget{before-we-calculate-the-a-that-maximizes-variance-let-us-observe-how-the-scalar-projections-on-a-random-direction-a-look-like.}{%
\subparagraph{\texorpdfstring{Before we calculate the \(a\) that
maximizes variance, let us observe how the scalar projections on a
random direction \(a\) look
like.}{Before we calculate the a that maximizes variance, let us observe how the scalar projections on a random direction a look like.}}\label{before-we-calculate-the-a-that-maximizes-variance-let-us-observe-how-the-scalar-projections-on-a-random-direction-a-look-like.}}

    \begin{Verbatim}[commandchars=\\\{\}]
{\color{incolor}In [{\color{incolor}8}]:} \PY{n}{a\PYZus{}rand} \PY{o}{=} \PY{n}{np}\PY{o}{.}\PY{n}{random}\PY{o}{.}\PY{n}{rand}\PY{p}{(}\PY{l+m+mi}{542}\PY{p}{,}\PY{l+m+mi}{1}\PY{p}{)} \PY{c+c1}{\PYZsh{}generate a random direction}
        \PY{n}{a\PYZus{}rand} \PY{o}{=} \PY{n}{a\PYZus{}rand}\PY{o}{/}\PY{n}{np}\PY{o}{.}\PY{n}{linalg}\PY{o}{.}\PY{n}{norm}\PY{p}{(}\PY{n}{a\PYZus{}rand}\PY{p}{)} \PY{c+c1}{\PYZsh{}we normalize the vector}
        \PY{n}{scores\PYZus{}rand} \PY{o}{=} \PY{n}{f}\PY{p}{(}\PY{n}{X}\PY{p}{,} \PY{n}{a\PYZus{}rand}\PY{p}{)} \PY{c+c1}{\PYZsh{}recall definition of f above}
        \PY{c+c1}{\PYZsh{} Now we visualize the scores along a\PYZus{}rand}
        \PY{n}{plt}\PY{o}{.}\PY{n}{scatter}\PY{p}{(}\PY{n}{scores\PYZus{}rand}\PY{p}{,} \PY{n}{np}\PY{o}{.}\PY{n}{zeros\PYZus{}like}\PY{p}{(}\PY{n}{scores\PYZus{}rand}\PY{p}{)}\PY{p}{,} \PY{n}{c}\PY{o}{=}\PY{n}{affiliations}\PY{p}{)}
        \PY{n}{plt}\PY{o}{.}\PY{n}{title}\PY{p}{(}\PY{l+s+s1}{\PYZsq{}}\PY{l+s+s1}{Projections along random direction}\PY{l+s+s1}{\PYZsq{}}\PY{p}{)}
        \PY{n}{plt}\PY{o}{.}\PY{n}{show}\PY{p}{(}\PY{p}{)}
        
        \PY{n+nb}{print}\PY{p}{(}\PY{l+s+s2}{\PYZdq{}}\PY{l+s+s2}{Variance along random direction: }\PY{l+s+s2}{\PYZdq{}}\PY{p}{,} \PY{n}{scores\PYZus{}rand}\PY{o}{.}\PY{n}{var}\PY{p}{(}\PY{p}{)}\PY{p}{)}
\end{Verbatim}


    \begin{center}
    \adjustimage{max size={0.9\linewidth}{0.9\paperheight}}{output_16_0.png}
    \end{center}
    { \hspace*{\fill} \\}
    
    \begin{Verbatim}[commandchars=\\\{\}]
Variance along random direction:  9.26745439089334

    \end{Verbatim}

    \hypertarget{note-here-that-projecting-along-the-random-vector-a_rand-does-not-explain-much-variance-at-all-it-is-clear-that-this-direction-does-not-give-us-any-information-about-the-senators-affiliations.}{%
\subparagraph{\texorpdfstring{Note here that projecting along the random
vector \(a\_{rand}\) does not explain much variance at all! It is clear
that this direction does not give us any information about the senators'
affiliations.}{Note here that projecting along the random vector a\textbackslash{}\_\{rand\} does not explain much variance at all! It is clear that this direction does not give us any information about the senators' affiliations.}}\label{note-here-that-projecting-along-the-random-vector-a_rand-does-not-explain-much-variance-at-all-it-is-clear-that-this-direction-does-not-give-us-any-information-about-the-senators-affiliations.}}

    \hypertarget{next-let-us-find-direction-a_1-that-maximizes-variance.-this-will-be-the-first-principal-component-of-x.}{%
\paragraph{Next let us find direction a\_1 that maximizes variance. This
will be the first principal component of
X.}\label{next-let-us-find-direction-a_1-that-maximizes-variance.-this-will-be-the-first-principal-component-of-x.}}

    \begin{Verbatim}[commandchars=\\\{\}]
{\color{incolor}In [{\color{incolor}14}]:} \PY{c+c1}{\PYZsh{}TODO: write code to get a\PYZus{}1, the first principal component of X(Note that shape of a\PYZus{}1 must be (542, 1)) }
         \PY{n}{pca} \PY{o}{=} \PY{n}{PCA}\PY{p}{(}\PY{n}{n\PYZus{}components}\PY{o}{=}\PY{l+m+mi}{1}\PY{p}{)}
         \PY{n}{pca}\PY{o}{.}\PY{n}{fit}\PY{p}{(}\PY{n}{X}\PY{p}{)}
         \PY{n}{a\PYZus{}1} \PY{o}{=} \PY{n}{pca}\PY{o}{.}\PY{n}{components\PYZus{}}\PY{o}{.}\PY{n}{reshape}\PY{p}{(}\PY{o}{\PYZhy{}}\PY{l+m+mi}{1}\PY{p}{,}\PY{p}{)} \PY{c+c1}{\PYZsh{}TODO replace this line with code to get a\PYZus{}1 that maximizes variance}
         \PY{c+c1}{\PYZsh{}Hint: the PCA packagle imported from sklearn.decomposition will be useful here. Look up the function}
         \PY{c+c1}{\PYZsh{}pca.fit() from its documentation}
         
         
         \PY{c+c1}{\PYZsh{}Next we compute scores along first principal component}
         \PY{n}{scores\PYZus{}a\PYZus{}1} \PY{o}{=} \PY{n}{f}\PY{p}{(}\PY{n}{X}\PY{p}{,} \PY{n}{a\PYZus{}1}\PY{p}{)} \PY{c+c1}{\PYZsh{}recall definition of f above}
         \PY{n}{plt}\PY{o}{.}\PY{n}{scatter}\PY{p}{(}\PY{n}{scores\PYZus{}a\PYZus{}1}\PY{p}{,} \PY{n}{np}\PY{o}{.}\PY{n}{zeros\PYZus{}like}\PY{p}{(}\PY{n}{scores\PYZus{}a\PYZus{}1}\PY{p}{)}\PY{p}{,} \PY{n}{c}\PY{o}{=}\PY{n}{affiliations}\PY{p}{)}
         \PY{n}{plt}\PY{o}{.}\PY{n}{title}\PY{p}{(}\PY{l+s+s1}{\PYZsq{}}\PY{l+s+s1}{Projections along first principal component}\PY{l+s+s1}{\PYZsq{}}\PY{p}{)}
         \PY{n}{plt}\PY{o}{.}\PY{n}{show}\PY{p}{(}\PY{p}{)}
         
         \PY{n+nb}{print}\PY{p}{(}\PY{l+s+s2}{\PYZdq{}}\PY{l+s+s2}{Variance along first principal component: }\PY{l+s+s2}{\PYZdq{}}\PY{p}{,} \PY{n}{scores\PYZus{}a\PYZus{}1}\PY{o}{.}\PY{n}{var}\PY{p}{(}\PY{p}{)}\PY{p}{)}
\end{Verbatim}


    \begin{center}
    \adjustimage{max size={0.9\linewidth}{0.9\paperheight}}{output_19_0.png}
    \end{center}
    { \hspace*{\fill} \\}
    
    \begin{Verbatim}[commandchars=\\\{\}]
Variance along first principal component:  149.74896507620747

    \end{Verbatim}

    \hypertarget{we-can-see-that-majority-of-the-blue-is-close-to-one-side-of-the-axis-and-red-is-close-to-the-other-side.-this-shows-that-the-first-principal-component-direction-explains-the-vote-spread-tends-to-align-with-party-affiliation.}{%
\subparagraph{We can see that majority of the blue is close to one side
of the axis and red is close to the other side. This shows that the
first principal component direction explains the vote spread tends to
align with party
affiliation.}\label{we-can-see-that-majority-of-the-blue-is-close-to-one-side-of-the-axis-and-red-is-close-to-the-other-side.-this-shows-that-the-first-principal-component-direction-explains-the-vote-spread-tends-to-align-with-party-affiliation.}}

    \hypertarget{part-b-comparison-to-party-averages}{%
\subsubsection{Part b) Comparison to party
averages}\label{part-b-comparison-to-party-averages}}

\hypertarget{building-on-the-observation-that-senators-vote-in-line-with-their-party-average-let-us-compute-variance-along-the-following-two-directions}{%
\subparagraph{Building on the observation that senators vote in line
with their party average let us compute variance along the following two
directions:}\label{building-on-the-observation-that-senators-vote-in-line-with-their-party-average-let-us-compute-variance-along-the-following-two-directions}}

a\_mean\_red: Unit vector along mean of rows corresponding to RED
senators\\
a\_mean\_blue: Unit vector along mean of rows corresponding to BLUE
senators

    \begin{Verbatim}[commandchars=\\\{\}]
{\color{incolor}In [{\color{incolor}15}]:} \PY{n}{aff} \PY{o}{=} \PY{n}{np}\PY{o}{.}\PY{n}{array}\PY{p}{(}\PY{n}{affiliations}\PY{p}{)}
         \PY{n}{mu\PYZus{}red} \PY{o}{=} \PY{n}{X}\PY{p}{[}\PY{n}{aff} \PY{o}{==} \PY{l+s+s1}{\PYZsq{}}\PY{l+s+s1}{Red}\PY{l+s+s1}{\PYZsq{}}\PY{p}{,} \PY{p}{:}\PY{p}{]}\PY{o}{.}\PY{n}{mean}\PY{p}{(}\PY{n}{axis}\PY{o}{=}\PY{l+m+mi}{0}\PY{p}{)} \PY{c+c1}{\PYZsh{}TODO Replace this line with mu\PYZus{}red (with shape (542,1)),the mean of rows of X}
         \PY{c+c1}{\PYZsh{}corresponding to Red senators as given by affiliations. }
         \PY{c+c1}{\PYZsh{}Hint: Print out affiliations and check what its entries are:}
         \PY{c+c1}{\PYZsh{} print(len(affiliations))}
         \PY{c+c1}{\PYZsh{} print(affiliations)}
         
         \PY{n+nb}{print}\PY{p}{(}\PY{n}{mu\PYZus{}red}\PY{o}{.}\PY{n}{shape}\PY{p}{)}
         \PY{n}{a\PYZus{}mean\PYZus{}red} \PY{o}{=} \PY{n}{mu\PYZus{}red}\PY{o}{/}\PY{n}{np}\PY{o}{.}\PY{n}{linalg}\PY{o}{.}\PY{n}{norm}\PY{p}{(}\PY{n}{mu\PYZus{}red}\PY{p}{)} \PY{c+c1}{\PYZsh{} normalize the vector }
         \PY{n}{scores\PYZus{}mean\PYZus{}red} \PY{o}{=} \PY{n}{f}\PY{p}{(} \PY{n}{X}\PY{p}{,} \PY{n}{a\PYZus{}mean\PYZus{}red}\PY{p}{)}
         \PY{n}{plt}\PY{o}{.}\PY{n}{scatter}\PY{p}{(}\PY{n}{scores\PYZus{}mean\PYZus{}red}\PY{p}{,} \PY{n}{np}\PY{o}{.}\PY{n}{zeros\PYZus{}like}\PY{p}{(}\PY{n}{scores\PYZus{}mean\PYZus{}red}\PY{p}{)}\PY{p}{,} \PY{n}{c}\PY{o}{=}\PY{n}{affiliations}\PY{p}{)}
         \PY{n}{plt}\PY{o}{.}\PY{n}{title}\PY{p}{(}\PY{l+s+s1}{\PYZsq{}}\PY{l+s+s1}{Projections along mean voting vector of red senators}\PY{l+s+s1}{\PYZsq{}}\PY{p}{)}
         \PY{n}{plt}\PY{o}{.}\PY{n}{show}\PY{p}{(}\PY{p}{)}
         
         \PY{n+nb}{print}\PY{p}{(}\PY{l+s+s2}{\PYZdq{}}\PY{l+s+s2}{Variance along mean voting vector of red senators: }\PY{l+s+s2}{\PYZdq{}}\PY{p}{,} \PY{n}{scores\PYZus{}mean\PYZus{}red}\PY{o}{.}\PY{n}{var}\PY{p}{(}\PY{p}{)}\PY{p}{)}
         
         \PY{c+c1}{\PYZsh{}Let us check angle between this and the first prinicpal component}
         \PY{n}{dot\PYZus{}product\PYZus{}red\PYZus{}a1} \PY{o}{=} \PY{n+nb}{float}\PY{p}{(}\PY{n}{np}\PY{o}{.}\PY{n}{dot}\PY{p}{(}\PY{n}{a\PYZus{}mean\PYZus{}red}\PY{o}{.}\PY{n}{T}\PY{p}{,} \PY{n}{a\PYZus{}1}\PY{p}{)}\PY{p}{)}
         \PY{n}{angle\PYZus{}red\PYZus{}a1} \PY{o}{=} \PY{n}{np}\PY{o}{.}\PY{n}{arccos}\PY{p}{(}\PY{n}{dot\PYZus{}product\PYZus{}red\PYZus{}a1}\PY{p}{)}\PY{o}{*}\PY{l+m+mi}{180}\PY{o}{/}\PY{n}{np}\PY{o}{.}\PY{n}{pi}
         
         \PY{n+nb}{print}\PY{p}{(}\PY{l+s+s2}{\PYZdq{}}\PY{l+s+s2}{Dot product of a\PYZus{}mean\PYZus{}red and a\PYZus{}1:}\PY{l+s+s2}{\PYZdq{}}\PY{p}{,} \PY{n}{dot\PYZus{}product\PYZus{}red\PYZus{}a1}\PY{p}{)}
         \PY{n+nb}{print}\PY{p}{(}\PY{l+s+s2}{\PYZdq{}}\PY{l+s+s2}{Angle between a\PYZus{}mean\PYZus{}red and a\PYZus{}1 in degrees:}\PY{l+s+s2}{\PYZdq{}}\PY{p}{,} \PY{n}{angle\PYZus{}red\PYZus{}a1}\PY{p}{)}
\end{Verbatim}


    \begin{Verbatim}[commandchars=\\\{\}]
(542,)

    \end{Verbatim}

    \begin{center}
    \adjustimage{max size={0.9\linewidth}{0.9\paperheight}}{output_22_1.png}
    \end{center}
    { \hspace*{\fill} \\}
    
    \begin{Verbatim}[commandchars=\\\{\}]
Variance along mean voting vector of red senators:  148.80699963205723
Dot product of a\_mean\_red and a\_1: -0.9965356912812978
Angle between a\_mean\_red and a\_1 in degrees: 175.22941782780276

    \end{Verbatim}

    \begin{Verbatim}[commandchars=\\\{\}]
{\color{incolor}In [{\color{incolor}16}]:} \PY{n}{mu\PYZus{}blue} \PY{o}{=} \PY{n}{X}\PY{p}{[}\PY{n}{aff} \PY{o}{==} \PY{l+s+s1}{\PYZsq{}}\PY{l+s+s1}{Blue}\PY{l+s+s1}{\PYZsq{}}\PY{p}{]}\PY{o}{.}\PY{n}{mean}\PY{p}{(}\PY{n}{axis}\PY{o}{=}\PY{l+m+mi}{0}\PY{p}{)} \PY{c+c1}{\PYZsh{}TODO Replace this line with mu\PYZus{}red (with shape (542,1)),the mean of rows of X }
         \PY{c+c1}{\PYZsh{}corresponding to Blue senators as given by affiliations. }
         \PY{c+c1}{\PYZsh{}Hint: Print out affiliations and check what its entries are:}
         \PY{c+c1}{\PYZsh{} print(len(affiliations))}
         \PY{c+c1}{\PYZsh{} print(affiliations)}
         
         \PY{n+nb}{print}\PY{p}{(}\PY{n}{mu\PYZus{}blue}\PY{o}{.}\PY{n}{shape}\PY{p}{)}
         
         \PY{n}{a\PYZus{}mean\PYZus{}blue} \PY{o}{=} \PY{n}{mu\PYZus{}blue}\PY{o}{/}\PY{n}{np}\PY{o}{.}\PY{n}{linalg}\PY{o}{.}\PY{n}{norm}\PY{p}{(}\PY{n}{mu\PYZus{}blue}\PY{p}{)} \PY{c+c1}{\PYZsh{} normalize the vector }
         \PY{n}{scores\PYZus{}mean\PYZus{}blue} \PY{o}{=} \PY{n}{f}\PY{p}{(} \PY{n}{X}\PY{p}{,} \PY{n}{a\PYZus{}mean\PYZus{}blue}\PY{p}{)}
         \PY{n}{plt}\PY{o}{.}\PY{n}{scatter}\PY{p}{(}\PY{n}{scores\PYZus{}mean\PYZus{}blue}\PY{p}{,} \PY{n}{np}\PY{o}{.}\PY{n}{zeros\PYZus{}like}\PY{p}{(}\PY{n}{scores\PYZus{}mean\PYZus{}blue}\PY{p}{)}\PY{p}{,} \PY{n}{c}\PY{o}{=}\PY{n}{affiliations}\PY{p}{)}
         \PY{n}{plt}\PY{o}{.}\PY{n}{title}\PY{p}{(}\PY{l+s+s1}{\PYZsq{}}\PY{l+s+s1}{Projections along mean voting vector of blue senators}\PY{l+s+s1}{\PYZsq{}}\PY{p}{)}
         \PY{n}{plt}\PY{o}{.}\PY{n}{show}\PY{p}{(}\PY{p}{)}
         
         \PY{n+nb}{print}\PY{p}{(}\PY{l+s+s2}{\PYZdq{}}\PY{l+s+s2}{Variance along mean voting vector of blue senators: }\PY{l+s+s2}{\PYZdq{}}\PY{p}{,} \PY{n}{scores\PYZus{}mean\PYZus{}blue}\PY{o}{.}\PY{n}{var}\PY{p}{(}\PY{p}{)}\PY{p}{)}
         
         \PY{c+c1}{\PYZsh{}Let us check angle between this and the first prinicpal component}
         \PY{n}{dot\PYZus{}product\PYZus{}blue\PYZus{}a1} \PY{o}{=} \PY{n+nb}{float}\PY{p}{(}\PY{n}{np}\PY{o}{.}\PY{n}{dot}\PY{p}{(}\PY{n}{a\PYZus{}mean\PYZus{}blue}\PY{o}{.}\PY{n}{T}\PY{p}{,} \PY{n}{a\PYZus{}1}\PY{p}{)}\PY{p}{)}
         \PY{n}{angle\PYZus{}blue\PYZus{}a1} \PY{o}{=} \PY{n}{np}\PY{o}{.}\PY{n}{arccos}\PY{p}{(}\PY{n}{dot\PYZus{}product\PYZus{}blue\PYZus{}a1}\PY{p}{)}\PY{o}{*}\PY{l+m+mi}{180}\PY{o}{/}\PY{n}{np}\PY{o}{.}\PY{n}{pi}
         
         \PY{n+nb}{print}\PY{p}{(}\PY{l+s+s2}{\PYZdq{}}\PY{l+s+s2}{Dot product of a\PYZus{}mean\PYZus{}blue and a\PYZus{}1:}\PY{l+s+s2}{\PYZdq{}}\PY{p}{,} \PY{n}{dot\PYZus{}product\PYZus{}blue\PYZus{}a1}\PY{p}{)}
         \PY{n+nb}{print}\PY{p}{(}\PY{l+s+s2}{\PYZdq{}}\PY{l+s+s2}{Angle between a\PYZus{}mean\PYZus{}blue and a\PYZus{}1 in degrees:}\PY{l+s+s2}{\PYZdq{}}\PY{p}{,} \PY{n}{angle\PYZus{}blue\PYZus{}a1}\PY{p}{)}
\end{Verbatim}


    \begin{Verbatim}[commandchars=\\\{\}]
(542,)

    \end{Verbatim}

    \begin{center}
    \adjustimage{max size={0.9\linewidth}{0.9\paperheight}}{output_23_1.png}
    \end{center}
    { \hspace*{\fill} \\}
    
    \begin{Verbatim}[commandchars=\\\{\}]
Variance along mean voting vector of blue senators:  148.9088414400461
Dot product of a\_mean\_blue and a\_1: 0.9969831227823032
Angle between a\_mean\_blue and a\_1 in degrees: 4.4516979833736166

    \end{Verbatim}

    \begin{Verbatim}[commandchars=\\\{\}]
{\color{incolor}In [{\color{incolor}17}]:} \PY{c+c1}{\PYZsh{}Finally let us compute angle between a\PYZus{}mean\PYZus{}red and a\PYZus{}mean\PYZus{}blue:}
         \PY{n}{dot\PYZus{}product\PYZus{}blue\PYZus{}red} \PY{o}{=} \PY{n+nb}{float}\PY{p}{(}\PY{n}{np}\PY{o}{.}\PY{n}{dot}\PY{p}{(}\PY{n}{a\PYZus{}mean\PYZus{}blue}\PY{o}{.}\PY{n}{T}\PY{p}{,} \PY{n}{a\PYZus{}mean\PYZus{}red}\PY{p}{)}\PY{p}{)}
         \PY{n}{angle\PYZus{}blue\PYZus{}red} \PY{o}{=} \PY{n}{np}\PY{o}{.}\PY{n}{arccos}\PY{p}{(}\PY{n}{dot\PYZus{}product\PYZus{}blue\PYZus{}red}\PY{p}{)}\PY{o}{*}\PY{l+m+mi}{180}\PY{o}{/}\PY{n}{np}\PY{o}{.}\PY{n}{pi}
         
         \PY{n+nb}{print}\PY{p}{(}\PY{l+s+s2}{\PYZdq{}}\PY{l+s+s2}{Dot product of a\PYZus{}mean\PYZus{}blue and mean\PYZus{}red:}\PY{l+s+s2}{\PYZdq{}}\PY{p}{,} \PY{n}{dot\PYZus{}product\PYZus{}blue\PYZus{}red}\PY{p}{)}
         \PY{n+nb}{print}\PY{p}{(}\PY{l+s+s2}{\PYZdq{}}\PY{l+s+s2}{Angle between a\PYZus{}mean\PYZus{}blue and mean\PYZus{}red in degrees:}\PY{l+s+s2}{\PYZdq{}}\PY{p}{,} \PY{n}{angle\PYZus{}blue\PYZus{}red}\PY{p}{)}
\end{Verbatim}


    \begin{Verbatim}[commandchars=\\\{\}]
Dot product of a\_mean\_blue and mean\_red: -0.9992350984093117
Angle between a\_mean\_blue and mean\_red in degrees: 177.75886458298191

    \end{Verbatim}

    \hypertarget{todo-fill-in-code-to-obtain-mu_red-and-mu_blue-in-the-cells-above.-comment-on-your-observations-about-how-the-party-averages-a_mean_red-and-a_mean_blue-are-related-to-the-first-principal-component-a_1.}{%
\subsubsection{\#TODO Fill in code to obtain mu\_red, and mu\_blue in
the cells above. Comment on your observations about how the party
averages (a\_mean\_red and a\_mean\_blue) are related to the first
principal component
(a\_1).}\label{todo-fill-in-code-to-obtain-mu_red-and-mu_blue-in-the-cells-above.-comment-on-your-observations-about-how-the-party-averages-a_mean_red-and-a_mean_blue-are-related-to-the-first-principal-component-a_1.}}

    We can see that the red and blue are in the opposite side of votes. The
blue votes are negative while the reds are positive.

    \hypertarget{part-c-computing-total-variance.-fill-in-code-in-cell-below-to-obtain-total-variance-along-first-two-principal-components.-refer-to-the-latex-file-for-more-details-on-the-question.}{%
\subsubsection{Part c) Computing total variance. Fill in code in cell
below to obtain total variance along first two principal components.
(Refer to the latex file for more details on the
question).}\label{part-c-computing-total-variance.-fill-in-code-in-cell-below-to-obtain-total-variance-along-first-two-principal-components.-refer-to-the-latex-file-for-more-details-on-the-question.}}

    \begin{Verbatim}[commandchars=\\\{\}]
{\color{incolor}In [{\color{incolor}22}]:} \PY{n}{X\PYZus{}bar} \PY{o}{=} \PY{n}{np}\PY{o}{.}\PY{n}{matmul}\PY{p}{(}\PY{n}{X}\PY{o}{.}\PY{n}{T}\PY{p}{,} \PY{n}{X}\PY{p}{)}\PY{o}{/}\PY{n}{n}
         
         \PY{n}{total\PYZus{}variance} \PY{o}{=} \PY{n}{PCA}\PY{p}{(}\PY{n}{n\PYZus{}components}\PY{o}{=}\PY{l+m+mi}{2}\PY{p}{)}\PY{o}{.}\PY{n}{fit}\PY{p}{(}\PY{n}{X\PYZus{}bar}\PY{p}{)}\PY{o}{.}\PY{n}{explained\PYZus{}variance\PYZus{}}\PY{o}{.}\PY{n}{sum}\PY{p}{(}\PY{p}{)} \PY{c+c1}{\PYZsh{}Replace with correct answer, the sum of largest two eigenvalues lambda\PYZus{}1, lambda\PYZus{}2 of X\PYZus{}bar}
         
         \PY{n+nb}{print}\PY{p}{(}\PY{l+s+s2}{\PYZdq{}}\PY{l+s+s2}{Total variance explained by first two principal components: }\PY{l+s+s2}{\PYZdq{}}\PY{p}{,} \PY{n}{total\PYZus{}variance}\PY{p}{)}
\end{Verbatim}


    \begin{Verbatim}[commandchars=\\\{\}]
Total variance explained by first two principal components:  42.02789225730074

    \end{Verbatim}

    \hypertarget{next-we-find-the-projection-onto-the-plane-spanned-by-the-first-two-principal-components}{%
\paragraph{Next we find the projection onto the plane spanned by the
first two principal
components}\label{next-we-find-the-projection-onto-the-plane-spanned-by-the-first-two-principal-components}}

    \begin{Verbatim}[commandchars=\\\{\}]
{\color{incolor}In [{\color{incolor}23}]:} \PY{n}{pca} \PY{o}{=} \PY{n}{PCA}\PY{p}{(}\PY{n}{n\PYZus{}components}\PY{o}{=}\PY{l+m+mi}{2}\PY{p}{)}
         \PY{n}{projected} \PY{o}{=} \PY{n}{pca}\PY{o}{.}\PY{n}{fit\PYZus{}transform}\PY{p}{(}\PY{n}{X}\PY{p}{)}
         
         \PY{n+nb}{print}\PY{p}{(}\PY{n}{projected}\PY{o}{.}\PY{n}{shape}\PY{p}{)}
         \PY{n}{plt}\PY{o}{.}\PY{n}{scatter}\PY{p}{(}\PY{n}{projected}\PY{p}{[}\PY{p}{:}\PY{p}{,} \PY{l+m+mi}{0}\PY{p}{]}\PY{p}{,} \PY{n}{projected}\PY{p}{[}\PY{p}{:}\PY{p}{,} \PY{l+m+mi}{1}\PY{p}{]}\PY{p}{,} \PY{n}{c}\PY{o}{=}\PY{n}{affiliations}\PY{p}{)}
         \PY{n}{plt}\PY{o}{.}\PY{n}{xlabel}\PY{p}{(}\PY{l+s+s1}{\PYZsq{}}\PY{l+s+s1}{a\PYZus{}1}\PY{l+s+s1}{\PYZsq{}}\PY{p}{)}
         \PY{n}{plt}\PY{o}{.}\PY{n}{ylabel}\PY{p}{(}\PY{l+s+s1}{\PYZsq{}}\PY{l+s+s1}{a\PYZus{}2}\PY{l+s+s1}{\PYZsq{}}\PY{p}{)}
         \PY{n}{plt}\PY{o}{.}\PY{n}{title}\PY{p}{(}\PY{l+s+s1}{\PYZsq{}}\PY{l+s+s1}{Projection on the plane spanned by first two principal components}\PY{l+s+s1}{\PYZsq{}}\PY{p}{)}
         \PY{n}{plt}\PY{o}{.}\PY{n}{show}\PY{p}{(}\PY{p}{)}
\end{Verbatim}


    \begin{Verbatim}[commandchars=\\\{\}]
(100, 2)

    \end{Verbatim}

    \begin{center}
    \adjustimage{max size={0.9\linewidth}{0.9\paperheight}}{output_30_1.png}
    \end{center}
    { \hspace*{\fill} \\}
    
    \hypertarget{part-d-finding-bills-that-are-the-mostleast-contentious}{%
\subsection{Part d) Finding bills that are the most/least
contentious}\label{part-d-finding-bills-that-are-the-mostleast-contentious}}

\hypertarget{approach-1-finding-variance-of-columns-of-x.-note-that-the-variance-of-column-j-can-be-viewed-as-the-variance-of-scores-along-the-direction-e_j-where-e_j-is-a-basis-vector-with-one-in-the-jth-entry-and-zero-elsewhere.}{%
\subsubsection{\texorpdfstring{Approach 1: Finding variance of columns
of X. Note that the variance of column \(j\) can be viewed as the
variance of scores along the direction \(e_j\), where \(e_j\) is a basis
vector with one in the \(j\)th entry and zero
elsewhere.}{Approach 1: Finding variance of columns of X. Note that the variance of column j can be viewed as the variance of scores along the direction e\_j, where e\_j is a basis vector with one in the jth entry and zero elsewhere.}}\label{approach-1-finding-variance-of-columns-of-x.-note-that-the-variance-of-column-j-can-be-viewed-as-the-variance-of-scores-along-the-direction-e_j-where-e_j-is-a-basis-vector-with-one-in-the-jth-entry-and-zero-elsewhere.}}

    \begin{Verbatim}[commandchars=\\\{\}]
{\color{incolor}In [{\color{incolor}24}]:} \PY{n}{list\PYZus{}variances} \PY{o}{=} \PY{n}{X}\PY{o}{.}\PY{n}{var}\PY{p}{(}\PY{l+m+mi}{0}\PY{p}{)} \PY{c+c1}{\PYZsh{} projects the standard basis in R\PYZca{}n for all bills; returns variances of each column}
         \PY{n}{bills} \PY{o}{=} \PY{n}{senator\PYZus{}df}\PY{p}{[}\PY{l+s+s1}{\PYZsq{}}\PY{l+s+s1}{bill\PYZus{}type bill\PYZus{}name bill\PYZus{}ID}\PY{l+s+s1}{\PYZsq{}}\PY{p}{]}\PY{o}{.}\PY{n}{values}
         
         \PY{n}{sorted\PYZus{}idx\PYZus{}variances} \PY{o}{=} \PY{n}{list\PYZus{}variances}\PY{o}{.}\PY{n}{argsort}\PY{p}{(}\PY{p}{)}\PY{p}{[}\PY{p}{:}\PY{p}{:}\PY{o}{\PYZhy{}}\PY{l+m+mi}{1}\PY{p}{]} \PY{c+c1}{\PYZsh{}TODO remove this line and replace it with code to }
         \PY{c+c1}{\PYZsh{}compute sorted\PYZus{}idx\PYZus{}variances: a np.array of shape (542,) containing integer entries that are indices}
         \PY{c+c1}{\PYZsh{} corresponding to decreasing order of variance of scores in list\PYZus{}variances. Hint: Use np.argsort()}
         \PY{c+c1}{\PYZsh{}Eg. If list\PYZus{}variances = [1,3,2,4], then sorted\PYZus{}idx\PYZus{}variances should be np.array([3,1,2,0])}
         
         
         
         \PY{n+nb}{print}\PY{p}{(}\PY{n}{sorted\PYZus{}idx\PYZus{}variances}\PY{o}{.}\PY{n}{shape}\PY{p}{)}
\end{Verbatim}


    \begin{Verbatim}[commandchars=\\\{\}]
(542,)

    \end{Verbatim}

    \hypertarget{todo-part-d-i-fill-in-code-to-compute-sorted_idx_variances-in-the-cell-above}{%
\subsubsection{\#TODO: Part d i) Fill in code to compute
sorted\_idx\_variances in the cell
above}\label{todo-part-d-i-fill-in-code-to-compute-sorted_idx_variances-in-the-cell-above}}

    \begin{Verbatim}[commandchars=\\\{\}]
{\color{incolor}In [{\color{incolor}25}]:} \PY{c+c1}{\PYZsh{} Retrive the bills with the top 5 variances and the lowest 5 variances}
         \PY{n}{top\PYZus{}5} \PY{o}{=} \PY{p}{[}\PY{n}{bills}\PY{p}{[}\PY{n}{sorted\PYZus{}idx\PYZus{}variances}\PY{p}{[}\PY{n}{i}\PY{p}{]}\PY{p}{]} \PY{k}{for} \PY{n}{i} \PY{o+ow}{in} \PY{n+nb}{range}\PY{p}{(}\PY{l+m+mi}{5}\PY{p}{)}\PY{p}{]}
         \PY{c+c1}{\PYZsh{} print(top\PYZus{}10)}
         \PY{n}{bot\PYZus{}5} \PY{o}{=} \PY{p}{[}\PY{n}{bills}\PY{p}{[}\PY{n}{sorted\PYZus{}idx\PYZus{}variances}\PY{p}{[}\PY{o}{\PYZhy{}}\PY{l+m+mi}{1}\PY{o}{\PYZhy{}}\PY{n}{i}\PY{p}{]}\PY{p}{]} \PY{k}{for} \PY{n}{i} \PY{o+ow}{in} \PY{n+nb}{range}\PY{p}{(}\PY{l+m+mi}{5}\PY{p}{)}\PY{p}{]}
         
         \PY{c+c1}{\PYZsh{} print(bot\PYZus{}10)}
         \PY{c+c1}{\PYZsh{}We look at voting pattern for bills with most and least variance using original non\PYZhy{}centered X matrix}
         \PY{n}{fig}\PY{p}{,} \PY{n}{axes} \PY{o}{=} \PY{n}{plt}\PY{o}{.}\PY{n}{subplots}\PY{p}{(}\PY{l+m+mi}{5}\PY{p}{,}\PY{l+m+mi}{2}\PY{p}{,} \PY{n}{figsize}\PY{o}{=}\PY{p}{(}\PY{l+m+mi}{15}\PY{p}{,}\PY{l+m+mi}{15}\PY{p}{)}\PY{p}{)} \PY{c+c1}{\PYZsh{} 1 plot to make things easier to see}
         \PY{k}{for} \PY{n}{i} \PY{o+ow}{in} \PY{n+nb}{range}\PY{p}{(}\PY{l+m+mi}{5}\PY{p}{)}\PY{p}{:} 
             \PY{n}{idx} \PY{o}{=} \PY{n}{sorted\PYZus{}idx\PYZus{}variances}\PY{p}{[}\PY{n}{i}\PY{p}{]}
         
             \PY{n}{X\PYZus{}red\PYZus{}c} \PY{o}{=} \PY{n}{X\PYZus{}original}\PY{p}{[}\PY{n}{np}\PY{o}{.}\PY{n}{array}\PY{p}{(}\PY{n}{affiliations}\PY{p}{)} \PY{o}{==} \PY{l+s+s1}{\PYZsq{}}\PY{l+s+s1}{Red}\PY{l+s+s1}{\PYZsq{}}\PY{p}{,}\PY{n}{idx}\PY{p}{]}
             \PY{n}{X\PYZus{}blue\PYZus{}c} \PY{o}{=} \PY{n}{X\PYZus{}original}\PY{p}{[}\PY{n}{np}\PY{o}{.}\PY{n}{array}\PY{p}{(}\PY{n}{affiliations}\PY{p}{)} \PY{o}{==} \PY{l+s+s1}{\PYZsq{}}\PY{l+s+s1}{Blue}\PY{l+s+s1}{\PYZsq{}}\PY{p}{,}\PY{n}{idx}\PY{p}{]}
             \PY{n}{X\PYZus{}yellow\PYZus{}c} \PY{o}{=} \PY{n}{X\PYZus{}original}\PY{p}{[}\PY{n}{np}\PY{o}{.}\PY{n}{array}\PY{p}{(}\PY{n}{affiliations}\PY{p}{)} \PY{o}{==} \PY{l+s+s1}{\PYZsq{}}\PY{l+s+s1}{Yellow}\PY{l+s+s1}{\PYZsq{}}\PY{p}{,}\PY{n}{idx}\PY{p}{]}
             
             \PY{n}{axes}\PY{p}{[}\PY{n}{i}\PY{p}{,}\PY{l+m+mi}{0}\PY{p}{]}\PY{o}{.}\PY{n}{hist}\PY{p}{(}\PY{p}{[}\PY{n}{X\PYZus{}red\PYZus{}c}\PY{p}{,} \PY{n}{X\PYZus{}blue\PYZus{}c}\PY{p}{,} \PY{n}{X\PYZus{}yellow\PYZus{}c}\PY{p}{]}\PY{p}{,} \PY{n}{color} \PY{o}{=} \PY{p}{[}\PY{l+s+s1}{\PYZsq{}}\PY{l+s+s1}{red}\PY{l+s+s1}{\PYZsq{}}\PY{p}{,} \PY{l+s+s1}{\PYZsq{}}\PY{l+s+s1}{blue}\PY{l+s+s1}{\PYZsq{}}\PY{p}{,} \PY{l+s+s1}{\PYZsq{}}\PY{l+s+s1}{yellow}\PY{l+s+s1}{\PYZsq{}}\PY{p}{]}\PY{p}{)}
             \PY{n}{axes}\PY{p}{[}\PY{n}{i}\PY{p}{,}\PY{l+m+mi}{0}\PY{p}{]}\PY{o}{.}\PY{n}{set\PYZus{}title}\PY{p}{(}\PY{n}{bills}\PY{p}{[}\PY{n}{idx}\PY{p}{]}\PY{p}{)}
         
         
         \PY{k}{for} \PY{n}{i} \PY{o+ow}{in} \PY{n+nb}{range}\PY{p}{(}\PY{l+m+mi}{1}\PY{p}{,}\PY{l+m+mi}{6}\PY{p}{)}\PY{p}{:} 
             \PY{n}{idx2} \PY{o}{=} \PY{n}{sorted\PYZus{}idx\PYZus{}variances}\PY{p}{[}\PY{o}{\PYZhy{}}\PY{n}{i}\PY{p}{]}
             \PY{n}{X\PYZus{}red\PYZus{}c2} \PY{o}{=} \PY{n}{X\PYZus{}original}\PY{p}{[}\PY{n}{np}\PY{o}{.}\PY{n}{array}\PY{p}{(}\PY{n}{affiliations}\PY{p}{)} \PY{o}{==} \PY{l+s+s1}{\PYZsq{}}\PY{l+s+s1}{Red}\PY{l+s+s1}{\PYZsq{}}\PY{p}{,}\PY{n}{idx2}\PY{p}{]}
             \PY{n}{X\PYZus{}blue\PYZus{}c2} \PY{o}{=} \PY{n}{X\PYZus{}original}\PY{p}{[}\PY{n}{np}\PY{o}{.}\PY{n}{array}\PY{p}{(}\PY{n}{affiliations}\PY{p}{)} \PY{o}{==} \PY{l+s+s1}{\PYZsq{}}\PY{l+s+s1}{Blue}\PY{l+s+s1}{\PYZsq{}}\PY{p}{,}\PY{n}{idx2}\PY{p}{]}
             \PY{n}{X\PYZus{}yellow\PYZus{}c2} \PY{o}{=} \PY{n}{X\PYZus{}original}\PY{p}{[}\PY{n}{np}\PY{o}{.}\PY{n}{array}\PY{p}{(}\PY{n}{affiliations}\PY{p}{)} \PY{o}{==} \PY{l+s+s1}{\PYZsq{}}\PY{l+s+s1}{Yellow}\PY{l+s+s1}{\PYZsq{}}\PY{p}{,}\PY{n}{idx2}\PY{p}{]}
             
             \PY{n}{axes}\PY{p}{[}\PY{n}{i}\PY{o}{\PYZhy{}}\PY{l+m+mi}{1}\PY{p}{,}\PY{l+m+mi}{1}\PY{p}{]}\PY{o}{.}\PY{n}{hist}\PY{p}{(}\PY{p}{[}\PY{n}{X\PYZus{}red\PYZus{}c2}\PY{p}{,} \PY{n}{X\PYZus{}blue\PYZus{}c2}\PY{p}{,} \PY{n}{X\PYZus{}yellow\PYZus{}c2}\PY{p}{]}\PY{p}{,} \PY{n}{color} \PY{o}{=} \PY{p}{[}\PY{l+s+s1}{\PYZsq{}}\PY{l+s+s1}{red}\PY{l+s+s1}{\PYZsq{}}\PY{p}{,} \PY{l+s+s1}{\PYZsq{}}\PY{l+s+s1}{blue}\PY{l+s+s1}{\PYZsq{}}\PY{p}{,} \PY{l+s+s1}{\PYZsq{}}\PY{l+s+s1}{yellow}\PY{l+s+s1}{\PYZsq{}}\PY{p}{]}\PY{p}{)}
             \PY{n}{axes}\PY{p}{[}\PY{n}{i}\PY{o}{\PYZhy{}}\PY{l+m+mi}{1}\PY{p}{,}\PY{l+m+mi}{1}\PY{p}{]}\PY{o}{.}\PY{n}{set\PYZus{}title}\PY{p}{(}\PY{n}{bills}\PY{p}{[}\PY{n}{idx2}\PY{p}{]}\PY{p}{)}
         
         \PY{n}{plt}\PY{o}{.}\PY{n}{subplots\PYZus{}adjust}\PY{p}{(}\PY{n}{hspace}\PY{o}{=}\PY{l+m+mf}{0.5}\PY{p}{,} \PY{n}{wspace} \PY{o}{=} \PY{l+m+mi}{1}\PY{p}{)}    
         \PY{n}{fig}\PY{o}{.}\PY{n}{suptitle}\PY{p}{(}\PY{l+s+s1}{\PYZsq{}}\PY{l+s+s1}{Most Variance \PYZhy{}\PYZhy{} Least Variance}\PY{l+s+s1}{\PYZsq{}}\PY{p}{,} \PY{n}{fontsize}\PY{o}{=}\PY{l+m+mi}{16}\PY{p}{)}
         \PY{n}{plt}\PY{o}{.}\PY{n}{show}\PY{p}{(}\PY{p}{)}
\end{Verbatim}


    \begin{center}
    \adjustimage{max size={0.9\linewidth}{0.9\paperheight}}{output_34_0.png}
    \end{center}
    { \hspace*{\fill} \\}
    
    \hypertarget{todo-part-d-ii-comment-on-how-the-voting-looks-like-for-bills-with-most-variance-and-bills-with-least-variance}{%
\subsubsection{\#TODO Part d ii) Comment on how the voting looks like
for bills with most variance and bills with least
variance}\label{todo-part-d-ii-comment-on-how-the-voting-looks-like-for-bills-with-most-variance-and-bills-with-least-variance}}

    Bills with most variance shows that 4 out of 5 bills are supported by
Red while only one is supported by Blue. Among them there is no bill
that both are sided together. However, with bills with the least
variance, they are all supported together by Red and Blue.

    \hypertarget{approach-2-we-find-the-projection-of-the-basis-vector-corresponding-to-each-bill-on-to-the-first-principal-components-and-choose-those-bills-with-highest-absolute-value-of-projections.-note-that-this-is-equivalent-to-choosing-bills-based-on-highest-absolute-values-of-a_1.}{%
\subsubsection{Approach 2: We find the projection of the basis vector
corresponding to each bill on to the first principal components and
choose those bills with highest absolute value of projections. Note that
this is equivalent to choosing bills based on highest absolute values of
a\_1.}\label{approach-2-we-find-the-projection-of-the-basis-vector-corresponding-to-each-bill-on-to-the-first-principal-components-and-choose-those-bills-with-highest-absolute-value-of-projections.-note-that-this-is-equivalent-to-choosing-bills-based-on-highest-absolute-values-of-a_1.}}

    \begin{Verbatim}[commandchars=\\\{\}]
{\color{incolor}In [{\color{incolor}26}]:} \PY{c+c1}{\PYZsh{} Recall that a\PYZus{}1\PYZus{}scores holds the projection onto the first principal component}
         \PY{n}{a\PYZus{}1\PYZus{}flat} \PY{o}{=} \PY{n}{np}\PY{o}{.}\PY{n}{ndarray}\PY{o}{.}\PY{n}{flatten}\PY{p}{(}\PY{n}{a\PYZus{}1}\PY{p}{)} \PY{c+c1}{\PYZsh{} first, flatten the a\PYZus{}1 of len 542}
         \PY{n}{abs\PYZus{}a\PYZus{}1} \PY{o}{=} \PY{n}{np}\PY{o}{.}\PY{n}{abs}\PY{p}{(}\PY{n}{a\PYZus{}1\PYZus{}flat}\PY{p}{)}
         
         \PY{n}{sorted\PYZus{}idxes} \PY{o}{=} \PY{n}{np}\PY{o}{.}\PY{n}{argsort}\PY{p}{(}\PY{o}{\PYZhy{}}\PY{n}{abs\PYZus{}a\PYZus{}1}\PY{p}{)} \PY{c+c1}{\PYZsh{}in decreasing order}
         \PY{n+nb}{print}\PY{p}{(}\PY{n}{sorted\PYZus{}idxes}\PY{o}{.}\PY{n}{shape}\PY{p}{)}
         
         \PY{n}{top\PYZus{}5\PYZus{}a1} \PY{o}{=} \PY{p}{[}\PY{n}{bills}\PY{p}{[}\PY{n}{sorted\PYZus{}idxes}\PY{p}{[}\PY{n}{i}\PY{p}{]}\PY{p}{]} \PY{k}{for} \PY{n}{i} \PY{o+ow}{in} \PY{n+nb}{range}\PY{p}{(}\PY{l+m+mi}{5}\PY{p}{)}\PY{p}{]}
         \PY{n}{bot\PYZus{}5\PYZus{}a1} \PY{o}{=} \PY{p}{[}\PY{n}{bills}\PY{p}{[}\PY{n}{sorted\PYZus{}idxes}\PY{p}{[}\PY{o}{\PYZhy{}}\PY{l+m+mi}{1}\PY{o}{\PYZhy{}}\PY{n}{i}\PY{p}{]}\PY{p}{]} \PY{k}{for} \PY{n}{i} \PY{o+ow}{in} \PY{n+nb}{range}\PY{p}{(}\PY{l+m+mi}{5}\PY{p}{)}\PY{p}{]}
         
         
         \PY{n}{fig}\PY{p}{,} \PY{n}{axes} \PY{o}{=} \PY{n}{plt}\PY{o}{.}\PY{n}{subplots}\PY{p}{(}\PY{l+m+mi}{5}\PY{p}{,}\PY{l+m+mi}{2}\PY{p}{,} \PY{n}{figsize}\PY{o}{=}\PY{p}{(}\PY{l+m+mi}{15}\PY{p}{,}\PY{l+m+mi}{15}\PY{p}{)}\PY{p}{)} \PY{c+c1}{\PYZsh{} 1 plot to make things easier to see}
         
         \PY{k}{for} \PY{n}{i} \PY{o+ow}{in} \PY{n+nb}{range}\PY{p}{(}\PY{l+m+mi}{5}\PY{p}{)}\PY{p}{:} 
             \PY{n}{idx} \PY{o}{=} \PY{n}{sorted\PYZus{}idxes}\PY{p}{[}\PY{n}{i}\PY{p}{]}
         
             \PY{n}{X\PYZus{}red\PYZus{}c} \PY{o}{=} \PY{n}{X\PYZus{}original}\PY{p}{[}\PY{n}{np}\PY{o}{.}\PY{n}{array}\PY{p}{(}\PY{n}{affiliations}\PY{p}{)} \PY{o}{==} \PY{l+s+s1}{\PYZsq{}}\PY{l+s+s1}{Red}\PY{l+s+s1}{\PYZsq{}}\PY{p}{,}\PY{n}{idx}\PY{p}{]}
             \PY{n}{X\PYZus{}blue\PYZus{}c} \PY{o}{=} \PY{n}{X\PYZus{}original}\PY{p}{[}\PY{n}{np}\PY{o}{.}\PY{n}{array}\PY{p}{(}\PY{n}{affiliations}\PY{p}{)} \PY{o}{==} \PY{l+s+s1}{\PYZsq{}}\PY{l+s+s1}{Blue}\PY{l+s+s1}{\PYZsq{}}\PY{p}{,}\PY{n}{idx}\PY{p}{]}
             \PY{n}{X\PYZus{}yellow\PYZus{}c} \PY{o}{=} \PY{n}{X\PYZus{}original}\PY{p}{[}\PY{n}{np}\PY{o}{.}\PY{n}{array}\PY{p}{(}\PY{n}{affiliations}\PY{p}{)} \PY{o}{==} \PY{l+s+s1}{\PYZsq{}}\PY{l+s+s1}{Yellow}\PY{l+s+s1}{\PYZsq{}}\PY{p}{,}\PY{n}{idx}\PY{p}{]}
             
             \PY{n}{axes}\PY{p}{[}\PY{n}{i}\PY{p}{,}\PY{l+m+mi}{0}\PY{p}{]}\PY{o}{.}\PY{n}{hist}\PY{p}{(}\PY{p}{[}\PY{n}{X\PYZus{}red\PYZus{}c}\PY{p}{,} \PY{n}{X\PYZus{}blue\PYZus{}c}\PY{p}{,} \PY{n}{X\PYZus{}yellow\PYZus{}c}\PY{p}{]}\PY{p}{,} \PY{n}{color} \PY{o}{=} \PY{p}{[}\PY{l+s+s1}{\PYZsq{}}\PY{l+s+s1}{red}\PY{l+s+s1}{\PYZsq{}}\PY{p}{,} \PY{l+s+s1}{\PYZsq{}}\PY{l+s+s1}{blue}\PY{l+s+s1}{\PYZsq{}}\PY{p}{,} \PY{l+s+s1}{\PYZsq{}}\PY{l+s+s1}{yellow}\PY{l+s+s1}{\PYZsq{}}\PY{p}{]}\PY{p}{)}
             \PY{n}{axes}\PY{p}{[}\PY{n}{i}\PY{p}{,}\PY{l+m+mi}{0}\PY{p}{]}\PY{o}{.}\PY{n}{set\PYZus{}title}\PY{p}{(}\PY{n}{bills}\PY{p}{[}\PY{n}{idx}\PY{p}{]}\PY{p}{)}
         
         
         \PY{k}{for} \PY{n}{i} \PY{o+ow}{in} \PY{n+nb}{range}\PY{p}{(}\PY{l+m+mi}{1}\PY{p}{,}\PY{l+m+mi}{6}\PY{p}{)}\PY{p}{:} 
             \PY{n}{idx2} \PY{o}{=} \PY{n}{sorted\PYZus{}idxes}\PY{p}{[}\PY{o}{\PYZhy{}}\PY{n}{i}\PY{p}{]}
             
             \PY{n}{X\PYZus{}red\PYZus{}c2} \PY{o}{=} \PY{n}{X\PYZus{}original}\PY{p}{[}\PY{n}{np}\PY{o}{.}\PY{n}{array}\PY{p}{(}\PY{n}{affiliations}\PY{p}{)} \PY{o}{==} \PY{l+s+s1}{\PYZsq{}}\PY{l+s+s1}{Red}\PY{l+s+s1}{\PYZsq{}}\PY{p}{,}\PY{n}{idx2}\PY{p}{]}
             \PY{n}{X\PYZus{}blue\PYZus{}c2} \PY{o}{=} \PY{n}{X\PYZus{}original}\PY{p}{[}\PY{n}{np}\PY{o}{.}\PY{n}{array}\PY{p}{(}\PY{n}{affiliations}\PY{p}{)} \PY{o}{==} \PY{l+s+s1}{\PYZsq{}}\PY{l+s+s1}{Blue}\PY{l+s+s1}{\PYZsq{}}\PY{p}{,}\PY{n}{idx2}\PY{p}{]}
             \PY{n}{X\PYZus{}yellow\PYZus{}c2} \PY{o}{=} \PY{n}{X\PYZus{}original}\PY{p}{[}\PY{n}{np}\PY{o}{.}\PY{n}{array}\PY{p}{(}\PY{n}{affiliations}\PY{p}{)} \PY{o}{==} \PY{l+s+s1}{\PYZsq{}}\PY{l+s+s1}{Yellow}\PY{l+s+s1}{\PYZsq{}}\PY{p}{,}\PY{n}{idx2}\PY{p}{]}
             
             \PY{n}{axes}\PY{p}{[}\PY{n}{i}\PY{o}{\PYZhy{}}\PY{l+m+mi}{1}\PY{p}{,}\PY{l+m+mi}{1}\PY{p}{]}\PY{o}{.}\PY{n}{hist}\PY{p}{(}\PY{p}{[}\PY{n}{X\PYZus{}red\PYZus{}c2}\PY{p}{,} \PY{n}{X\PYZus{}blue\PYZus{}c2}\PY{p}{,} \PY{n}{X\PYZus{}yellow\PYZus{}c2}\PY{p}{]}\PY{p}{,} \PY{n}{color} \PY{o}{=} \PY{p}{[}\PY{l+s+s1}{\PYZsq{}}\PY{l+s+s1}{red}\PY{l+s+s1}{\PYZsq{}}\PY{p}{,} \PY{l+s+s1}{\PYZsq{}}\PY{l+s+s1}{blue}\PY{l+s+s1}{\PYZsq{}}\PY{p}{,} \PY{l+s+s1}{\PYZsq{}}\PY{l+s+s1}{yellow}\PY{l+s+s1}{\PYZsq{}}\PY{p}{]}\PY{p}{)}
             \PY{n}{axes}\PY{p}{[}\PY{n}{i}\PY{o}{\PYZhy{}}\PY{l+m+mi}{1}\PY{p}{,}\PY{l+m+mi}{1}\PY{p}{]}\PY{o}{.}\PY{n}{set\PYZus{}title}\PY{p}{(}\PY{n}{bills}\PY{p}{[}\PY{n}{idx2}\PY{p}{]}\PY{p}{)}
         
         \PY{n}{plt}\PY{o}{.}\PY{n}{subplots\PYZus{}adjust}\PY{p}{(}\PY{n}{hspace}\PY{o}{=}\PY{l+m+mf}{0.5}\PY{p}{,} \PY{n}{wspace} \PY{o}{=} \PY{l+m+mi}{1}\PY{p}{)}    
         \PY{n}{fig}\PY{o}{.}\PY{n}{suptitle}\PY{p}{(}\PY{l+s+s1}{\PYZsq{}}\PY{l+s+s1}{Highest abs a\PYZus{}1 \PYZhy{}\PYZhy{} Lowest abs a\PYZus{}1}\PY{l+s+s1}{\PYZsq{}}\PY{p}{,} \PY{n}{fontsize}\PY{o}{=}\PY{l+m+mi}{16}\PY{p}{)}
         \PY{n}{plt}\PY{o}{.}\PY{n}{show}\PY{p}{(}\PY{p}{)}
\end{Verbatim}


    \begin{Verbatim}[commandchars=\\\{\}]
(542,)

    \end{Verbatim}

    \begin{center}
    \adjustimage{max size={0.9\linewidth}{0.9\paperheight}}{output_38_1.png}
    \end{center}
    { \hspace*{\fill} \\}
    
    \hypertarget{todo-part-d-iii-comment-on-how-the-voting-looks-like-for-bills-with-highest-and-lowest-absolute-values-of-a_1.}{%
\subsubsection{\#TODO Part d iii) Comment on how the voting looks like
for bills with highest and lowest absolute values of
a\_1.}\label{todo-part-d-iii-comment-on-how-the-voting-looks-like-for-bills-with-highest-and-lowest-absolute-values-of-a_1.}}

    Bills with the highest divide Red and Blue with no overlaps while with
the least absolute values, they are all alost equally overlapped and
supported together by Red and Blue, just like the bills with the least
variance.

    \hypertarget{next-let-us-compare-the-bills-found-by-the-two-approaches}{%
\paragraph{Next let us compare the bills found by the two
approaches}\label{next-let-us-compare-the-bills-found-by-the-two-approaches}}

    \begin{Verbatim}[commandchars=\\\{\}]
{\color{incolor}In [{\color{incolor}27}]:} \PY{c+c1}{\PYZsh{} The bills that are the same in both the top and bottom 10 using these different methods:}
         \PY{n+nb}{print}\PY{p}{(}\PY{l+s+s1}{\PYZsq{}}\PY{l+s+s1}{Number of common bills in top:}\PY{l+s+s1}{\PYZsq{}}\PY{p}{,} \PY{n+nb}{len}\PY{p}{(}\PY{n}{np}\PY{o}{.}\PY{n}{intersect1d}\PY{p}{(}\PY{n}{top\PYZus{}5} \PY{p}{,}\PY{n}{top\PYZus{}5\PYZus{}a1}\PY{p}{)}\PY{p}{)}\PY{p}{)}
         \PY{n+nb}{print}\PY{p}{(}\PY{l+s+s1}{\PYZsq{}}\PY{l+s+s1}{Number of common bills in bottom :}\PY{l+s+s1}{\PYZsq{}}\PY{p}{,} \PY{n+nb}{len}\PY{p}{(}\PY{n}{np}\PY{o}{.}\PY{n}{intersect1d}\PY{p}{(}\PY{n}{bot\PYZus{}5} \PY{p}{,}\PY{n}{bot\PYZus{}5\PYZus{}a1}\PY{p}{)}\PY{p}{)}\PY{p}{)}
\end{Verbatim}


    \begin{Verbatim}[commandchars=\\\{\}]
Number of common bills in top: 0
Number of common bills in bottom : 1

    \end{Verbatim}

    \hypertarget{todo-part-d-iv-are-the-bills-in-the-two-approaches-the-same-what-do-you-think-is-the-reason-for-the-difference}{%
\subsubsection{\#TODO Part d iv) Are the bills in the two approaches the
same? What do you think is the reason for the
difference?}\label{todo-part-d-iv-are-the-bills-in-the-two-approaches-the-same-what-do-you-think-is-the-reason-for-the-difference}}

    No, they are not the same because they are not projected in the same
line.

    \hypertarget{part-e-finally-we-will-look-at-the-scores-for-senators-along-the-first-principal-direction-and-make-the-following-classifications-for-senators}{%
\subsubsection{Part e) Finally, we will look at the scores for senators
along the first principal direction and make the following
classifications for
senators:}\label{part-e-finally-we-will-look-at-the-scores-for-senators-along-the-first-principal-direction-and-make-the-following-classifications-for-senators}}

\begin{enumerate}
\def\labelenumi{\alph{enumi})}
\tightlist
\item
  Senators with the top 10 most positive scores and top 10 most negative
  scores are classified as most "extreme''.
\item
  Senators with the 20 scores closest to 0 are classified as least
  ``extreme''.
\end{enumerate}

    \hypertarget{most-extreme-senators}{%
\paragraph{Most extreme senators}\label{most-extreme-senators}}

    \begin{Verbatim}[commandchars=\\\{\}]
{\color{incolor}In [{\color{incolor}35}]:} \PY{n}{senators} \PY{o}{=} \PY{n}{senator\PYZus{}df}\PY{o}{.}\PY{n}{columns}\PY{o}{.}\PY{n}{values}\PY{p}{[}\PY{l+m+mi}{3}\PY{p}{:}\PY{p}{]}
         
         \PY{n}{senator\PYZus{}scores} \PY{o}{=} \PY{n}{f}\PY{p}{(}\PY{n}{X}\PY{p}{,}\PY{n}{a\PYZus{}1}\PY{p}{)}
         \PY{n}{complete\PYZus{}sort\PYZus{}indices} \PY{o}{=} \PY{n}{np}\PY{o}{.}\PY{n}{argsort}\PY{p}{(}\PY{n}{senator\PYZus{}scores}\PY{p}{)}
         
         \PY{n}{sort\PYZus{}indices} \PY{o}{=} \PY{n}{np}\PY{o}{.}\PY{n}{hstack}\PY{p}{(}\PY{p}{[}\PY{n}{complete\PYZus{}sort\PYZus{}indices}\PY{p}{[}\PY{p}{:}\PY{l+m+mi}{10}\PY{p}{]}\PY{p}{,} \PY{n}{complete\PYZus{}sort\PYZus{}indices}\PY{p}{[}\PY{o}{\PYZhy{}}\PY{l+m+mi}{11}\PY{p}{:}\PY{o}{\PYZhy{}}\PY{l+m+mi}{1}\PY{p}{]}\PY{p}{]}\PY{p}{)}
         \PY{n}{senators\PYZus{}sorted} \PY{o}{=} \PY{n}{senators}\PY{p}{[}\PY{n}{sort\PYZus{}indices}\PY{p}{]}
         \PY{n}{senator\PYZus{}scores\PYZus{}sorted} \PY{o}{=} \PY{n}{senator\PYZus{}scores}\PY{p}{[}\PY{n}{sort\PYZus{}indices}\PY{p}{]}
         \PY{n}{affiliations} \PY{o}{=} \PY{n}{np}\PY{o}{.}\PY{n}{array}\PY{p}{(}\PY{n}{affiliations}\PY{p}{)}
         \PY{n}{affiliations\PYZus{}sorted} \PY{o}{=} \PY{n}{affiliations}\PY{p}{[}\PY{n}{sort\PYZus{}indices}\PY{p}{]}
         
         \PY{n}{plt}\PY{o}{.}\PY{n}{barh}\PY{p}{(}\PY{n}{y} \PY{o}{=} \PY{n}{senators\PYZus{}sorted}\PY{p}{,} \PY{n}{width} \PY{o}{=} \PY{n}{senator\PYZus{}scores\PYZus{}sorted}\PY{p}{,} \PY{n}{color} \PY{o}{=} \PY{n}{affiliations\PYZus{}sorted}\PY{p}{)}
         \PY{n}{plt}\PY{o}{.}\PY{n}{title}\PY{p}{(}\PY{l+s+s1}{\PYZsq{}}\PY{l+s+s1}{Scores of the most extreme Red and Blue senators}\PY{l+s+s1}{\PYZsq{}}\PY{p}{)}
         \PY{n}{plt}\PY{o}{.}\PY{n}{show}\PY{p}{(}\PY{p}{)}
\end{Verbatim}


    \begin{center}
    \adjustimage{max size={0.9\linewidth}{0.9\paperheight}}{output_47_0.png}
    \end{center}
    { \hspace*{\fill} \\}
    
    \hypertarget{least-extreme-senators}{%
\paragraph{Least extreme senators}\label{least-extreme-senators}}

    \begin{Verbatim}[commandchars=\\\{\}]
{\color{incolor}In [{\color{incolor}37}]:} \PY{n}{senator\PYZus{}scores} \PY{o}{=} \PY{n}{f}\PY{p}{(}\PY{n}{X}\PY{p}{,}\PY{n}{a\PYZus{}1}\PY{p}{)}
         \PY{c+c1}{\PYZsh{} print(np.sort(np.abs(senator\PYZus{}scores)))}
         \PY{n}{complete\PYZus{}sort\PYZus{}indices} \PY{o}{=} \PY{n}{np}\PY{o}{.}\PY{n}{argsort}\PY{p}{(}\PY{n}{np}\PY{o}{.}\PY{n}{abs}\PY{p}{(}\PY{n}{senator\PYZus{}scores}\PY{p}{)}\PY{p}{)}\PY{p}{[}\PY{p}{:}\PY{l+m+mi}{20}\PY{p}{]}
         
         \PY{n}{senator\PYZus{}scores\PYZus{}le}\PY{o}{=} \PY{n}{senator\PYZus{}scores}\PY{p}{[}\PY{n}{complete\PYZus{}sort\PYZus{}indices}\PY{p}{]}
         \PY{n}{senators\PYZus{}le} \PY{o}{=} \PY{n}{senators}\PY{p}{[}\PY{n}{complete\PYZus{}sort\PYZus{}indices}\PY{p}{]}
         \PY{n}{affiliations} \PY{o}{=} \PY{n}{np}\PY{o}{.}\PY{n}{array}\PY{p}{(}\PY{n}{affiliations}\PY{p}{)}
         \PY{n}{affiliations\PYZus{}le} \PY{o}{=} \PY{n}{affiliations}\PY{p}{[}\PY{n}{complete\PYZus{}sort\PYZus{}indices}\PY{p}{]}
         \PY{n}{sort\PYZus{}indices} \PY{o}{=} \PY{n}{np}\PY{o}{.}\PY{n}{argsort}\PY{p}{(}\PY{n}{senator\PYZus{}scores\PYZus{}le}\PY{p}{)}
         \PY{n}{senators\PYZus{}sorted} \PY{o}{=} \PY{n}{senators\PYZus{}le}\PY{p}{[}\PY{n}{sort\PYZus{}indices}\PY{p}{]}
         \PY{n}{senator\PYZus{}scores\PYZus{}sorted} \PY{o}{=} \PY{n}{senator\PYZus{}scores\PYZus{}le}\PY{p}{[}\PY{n}{sort\PYZus{}indices}\PY{p}{]}
         \PY{n}{affiliations\PYZus{}sorted} \PY{o}{=} \PY{n}{affiliations\PYZus{}le}\PY{p}{[}\PY{n}{sort\PYZus{}indices}\PY{p}{]}
         
         \PY{n}{plt}\PY{o}{.}\PY{n}{barh}\PY{p}{(}\PY{n}{y} \PY{o}{=} \PY{n}{senators\PYZus{}sorted}\PY{p}{,} \PY{n}{width} \PY{o}{=} \PY{n}{senator\PYZus{}scores\PYZus{}sorted}\PY{p}{,} \PY{n}{color} \PY{o}{=} \PY{n}{affiliations\PYZus{}sorted}\PY{p}{)}
         \PY{n}{plt}\PY{o}{.}\PY{n}{title}\PY{p}{(}\PY{l+s+s1}{\PYZsq{}}\PY{l+s+s1}{Scores of the least extreme Red and Blue senators}\PY{l+s+s1}{\PYZsq{}}\PY{p}{)}
         \PY{n}{plt}\PY{o}{.}\PY{n}{show}\PY{p}{(}\PY{p}{)}
\end{Verbatim}


    \begin{center}
    \adjustimage{max size={0.9\linewidth}{0.9\paperheight}}{output_49_0.png}
    \end{center}
    { \hspace*{\fill} \\}
    
    \hypertarget{todo-comment-on-the-sign-of-scores-vs-party-affiliations.}{%
\subsubsection{\#TODO Comment on the sign of scores vs party
affiliations.}\label{todo-comment-on-the-sign-of-scores-vs-party-affiliations.}}

    We can see that Red senators have more negative sign of scores than the
blues. Even the red senators with the least extreme show more negative
scores than the blue senators and there are more red senators with the
least extreme scores.


    % Add a bibliography block to the postdoc
    
    
    
    \end{document}
